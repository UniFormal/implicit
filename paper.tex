\documentclass[orivec]{llncs}

\usepackage{graphicx,url,wrapfig,amsmath,amssymb}

\usepackage{comment}
% flip these to add module expressions; if so, some manual changes to intro and grammars are also necessary
\excludecomment{modexp}
\includecomment{nomodexp}
% include unions as an example of module expressions
\excludecomment{union}

\usepackage[show]{ed} %for ednotes

\usepackage{macros/sembracket}
\usepackage{macros/mytikz}
\usepackage{macros/basics}
\usepackage{local}

\usepackage{listings}

\lstdefinelanguage{mmt}{
	morekeywords={theory,namespace,rule,role,import,implicit,definitional,renaming},
	morecomment=[l][\color{Gray}]{//},
	% morecomment={[s]{//}{❚},[s]{//}{❙},[s]{//}{❘}}
}
\lstnewenvironment{mmtcode}[1][language=mmt]{\lstset{language=mmt,keywordstyle=\bfseries,basicstyle=\ttfamily,mathescape,columns=fixed,#1}}{}


\setcounter{tocdepth}{3}
\usepackage[bookmarks,linkcolor=red,citecolor=blue,urlcolor=gray,colorlinks,breaklinks,bookmarksopen,bookmarksnumbered]{hyperref}

\setlength{\hfuzz}{3pt}
\hbadness=10001 %make box warning less strict

\pagestyle{plain}

\begin{document}
\title{Structuring Theories with Implicit Morphisms}
\author{Florian Rabe$^{1,2}$ \and Dennis M\"uller$^2$}
\institute{LRI Paris \and FAU Erlangen-Nuremberg}
\maketitle

\begin{abstract}
We introduce \emph{implicit} morphisms as a concept in formal systems based on theories and theory morphisms.
The idea is that there may be at most one implicit morphism from a theory $S$ to a theory $T$, and if $S$-expressions are used in $T$ their semantics is obtained by automatically inserting the implicit morphism.
The practical appeal of implicit morphisms is that they hit the sweet-spot of being extremely simple to understand and implement while significantly helping with structuring large collections of theories.
Concrete applications include elegantly identifying isomorphic theories and extending theories with definitions and theorems as well as efficiently building and maintaining large, fine-granular, and heterogeneous hierarchies of theories.
\end{abstract}


\section{Introduction}
\paragraph{Motivation}
Theory morphisms have proved an essential tool for managing collections of theories in logics and related formal systems.
They can be used to structure theories and build large theories modularly from small components or to relate different theories to each other \cite{asl,devgraphs,littletheories}.
Areas in which tools based on theories and theory morphisms have been developed include specification \cite{obj,hets}, rewriting \cite{maude}, theorem proving \cite{imps,isabelle_locales}, and knowledge representation \cite{RK:mmt:10}.
Closely related concepts are used in both object-oriented (\emph{classes}) and functional (\emph{type classes}) programming languages.

%The latter of these tools is our \mmt system.
%We use \mmt as both a theoretical framework and a concrete implementation of our results.
%But our ideas are independent of \mmt and can be transferred very easily to any other language using theory morphisms.

These systems usually use a logic $L$ for the low-level formalization of domain knowledge, and a diagram $D$ in the category of $L$-theories and $L$-morphisms for the high-level structure of large bodies of knowledge.

\begin{wrapfigure}{r}{5.5cm}
\vspace{-3em}
\begin{tikzpicture}
\node (M) at (0,0) {$\cn{Monoid}$};
\node (G) at (-2,-2) {$\cn{Group}$};
\node (D) at (2,-2) {$\cn{DivGroup}$};
\draw[mono](M) -- (G);
\draw[arrow](G) to[out=30,in=150] node[above] {$\cn{G2DG}$} (D);
\draw[arrow](D) to[out=-150,in=-30] node[above] {$\cn{DG2G}$} (G);
\end{tikzpicture}
\vspace{-3.5em}
\end{wrapfigure}

For example, a document might reference an existing theory $\cn{Monoid}$, define a new theory $\cn{Group}$ that extends $\cn{Monoid}$, define a theory $\cn{DivGroup}$ (providing an alternative formulation of groups based on the division operation), and then define two theory morphisms $\cn{G2DG}:\cn{Group}\darr \cn{DivGroup}:\cn{DG2G}$ that witness an isomorphism between these theories.
This would result in the diagram on the right.%
\footnote{Note that we use the syntactic direction for the arrows, e.g., an arrow $m:S\to T$ states that any $S$-expression $E$ (e.g., a sort, term, formula, or proof) can be translated to a $T$-expression $m(E)$. Models are translated in the opposite direction.}
%Crucially, $m(-)$ preserves typing and provability.

The key idea behind implicit morphisms is very simple:
We maintain an additional diagram $I$, which is a commutative subdiagram of $D$ and whose morphisms we call \emph{implicit}.
The condition of commutativity guarantees that $I$ has at most one morphism $i$ from theory $S$ to theory $T$, in which case we write $\ipc{i}{S}{T}$.
Commutativity makes the following language extension well-defined: if $\ipc{i}{S}{T}$, then any identifier $c$ that is visible to $S$ may also be used in $T$-expressions, with the semantics being that $c$ abbreviates $i(c)$.
For example, in the diagram above, we may choose to label \cn{DG2G} implicit.
Immediately, every abbreviation or theorem that we have formulated in the theory \cn{DivGroup} becomes available for use in \cn{Group} without any syntactic overhead.
We can even label \cn{G2DG} implicit as well if we prove the isomorphism property to ensure that $I$ remains commutative, thus capturing the mathematical intuition that \cn{Group} and \cn{DivGroup} are just different formalizations of the same concept.
While these morphisms must be labeled manually, any inclusion morphism like the one from \cn{Monoid} to \cn{Group} can be made implicit automatically.

\paragraph{Contribution}
At the highest level, our contribution is the observation that implicit morphisms form a sweet spot of a very simple language feature that has substantial practical uses.
We recommend using implicit morphisms in all theory morphism--based formalisms.
More concretely, we present a formal system for developing structured theories with implicit morphisms.
Our starting point is the \mmt language \cite{RK:mmt:10}, which already provides a very general setting for defining and working with theories and morphisms.
\mmt is logic-independent, i.e., allows embedding a large variety of declarative languages (logics, type-theories, etc.).
Therefore, all our results can be directly applied to any language $L$ represented in \mmt or easily transferred to dedicated implementations of $L$.

We describe several example applications of implicit morphisms in detail: the identification of isomorphic theories, definitional extensions of theories, building large hierarchies of theories with many rarely used intermediate theories, seamlessly moving theories across logic morphisms, and transparently refactoring theory hierarchies.

\paragraph{Previous Work}
Implicit morphism were first conceived by Rabe in 2010 and implemented as part of the Twelf system \cite{twelf} (which implements the dependent type theory LF).
The theory behind this implementation was never written up and not published.
But the implementation already scaled well, and implicit morphisms were used in the LATIN logic atlas \cite{CHKMR:latinabs:11} built by Rabe and others in 2009--2012.
The LATIN atlas already has around a 1000 theories and atomic morphisms, and about 50 of the latter are marked as implicit.
It also has a few hundred inclusions, each of which induces another implicit morphism.

Since then, MMT has been developed, and
\begin{compactitem}
 \item implicit morphisms were generalized from LF to the logic-independent level of MMT,
 \item their theory was worked out,
 \item they were reimplemented from scratch as a part of MMT.
\end{compactitem}
MMT is backwards-compatible with Twelf, and the LATIN atlas including its implicit morphisms can be used from within MMT.
The present paper introduces these results in their final, most elegant form.

\paragraph{Overview}
In Sect.~\ref{sec:mmt}, we present the syntax and semantics of \mmt.
Even though the \mmt language is not new, our presentation is an entirely novel contribution in itself: it is much simpler and more elegant than the original one in \cite{RK:mmt:10}.
%Based on this extensible definition, we can introduce several additional structuring mechanisms.
%Many of these have already been implemented in the \mmt tool but were not formally defined --- a formal definition of their semantics as a part of the \mmt language had previously proved too complicated to spell out elegantly.
%With our reformulation, this is now not only possible but very easy.
%As an example, we define include declarations, the most important special case of implicit morphisms.
%\medskip
%
%Secondly, we describe implicit morphisms as a novel feature of \mmt.
Crucially, this increase in simplicity allows spelling out the syntax and semantics of implicit morphisms, which we do in Sect.~\ref{sec:impl}, within a few pages.
In Sect.~\ref{sec:appl}, we present applications.
Finally we discuss related and future work in Sect.~\ref{sec:conc}.

%In fact, implicit morphisms work so well that we have refactored the \mmt tool in such a way that implicit morphisms are now more primitive than inclusion morphisms.
%The semantics of inclusion morphisms is obtained by saying that inclusions are implicit morphisms that map all identifiers to themselves.
%Even the fundamental property that a theory may reference its own identifiers is then just a consequence of the fact that all identity morphisms are implicit.
%Therefore, surprisingly, adding the new feature of implicit morphisms to the \mmt kernel has made its design much simpler.


\section{Theories and Theory Morphisms}\label{sec:mmt}
\ednote{@Dennis: can you formally work through the example from the intro in here? That would be several example environments for (i) after introducing the syntax of theories, the theory Group and DivGroup (one in detail, one sketched), (ii) after introducing morphisms the two isomorphisms (one in detail, one sketched), (iii) after introducing includes, the refactoring of Group using an include, (iv) examples for how the refactored Group is flattened}

\subsection{Overview}

\begin{modexp}
\paragraph{Modularity in MMT}
As much as possible, theories and morphisms are treated uniformly, and we use the word \textbf{module} to refer to them collectively.
\mmt allows constructing large modules from small ones in two ways:
\begin{compactitem}
  \item \textbf{structured modules} contain declarations that import other, previously defined ones,
  \item \textbf{module expressions} are anonymous expressions (akin to formulas, terms, etc.) that denote modules.
\end{compactitem}

These two kinds of constructions have very different advantages but the same expressivity.
For example, we can construct the theory of commutative groups as the structured theory $\cn{CommutativeGroup}=\{\icl{Group},\icl{CommutativeMagma}\}$ or as the theory expression $\cn{Group}\cup\cn{CommutativeMagma}$.
The former is more practical when building named toplevel theories in software systems because each include is a separate declaration that can be parsed, checked, and flattened individually.
While negligible in this example, this becomes relevant quickly in large theories with complex structure.
The latter is more practical when building anonymous theory that are only needed temporarily.
In software systems, they also allow reusing operations like equality and $\lambda$-abstraction that already be implemented at the level of the base logic.
Building large morphism is as important as building large theories: for example, we want to be able to build morphisms out of $\cn{CommutativeGroup}$ by combining compatible morphisms out of $\cn{Group}$ and $\cn{CommutativeMagma}$.
\end{modexp}

\paragraph{Flat Modules}
\mmt provides formal syntax for both structured modules and module expressions and defines their semantics via \textbf{flattening}, which defines for every module $M$ a flat module $\flt{M}$.

\textbf{Flat theories} are lists of declarations $c:E[=e]$ where $E$ and $e$ are expression, and the latter is optional.
We write $\dom{T}$ for the set of constant identifiers $c$ in $\flt{T}$ and $\Exp{T}$ for the set of closed expressions using only the symbols $c\in\dom{T}$ (see below for the definition of well-formed expressions).
Constant declarations subsume virtually all basic declarations common in formal systems such as type/function/predicate symbols, axioms, theorems, inference rules, etc.
In particular, theorems can be represented via the propositions-as-types correspondence as declarations $c:F=P$, which establish theorem $F$ via proof $P$.
Similarly, \mmt expressions subsume virtually all objects common in formal systems such as terms, types, formulas, proofs.

Individual formal languages arise as fragments of \mmt: they single out the well-formed expressions by defining the two \mmt-\textbf{judgments} $\der_T e:e'$ (typing) and $\der_T e\equiv e'$ (equality) for every theory $T$ and $e,e'\in\Exp{T}$.
%This is already sufficient to define logical theorems as those formulas $F$ for which there is an object $p$ (the proof) such that $\der_T p:F$.
The details can be found in \cite{rabe:howto:14}.

\textbf{Flat morphisms} from a theory $S$ to a theory $T$ are lists of assignments $c:=e$ where $c\in\dom{S}$ and $e\in\Exp{T}$.
Every morphism $M$ induces a \textbf{homomorphic extension} $M(-):\Exp{S}\to\Exp{T}$, which replaces every $c\in\dom{S}$ in an $S$-expression with the $T$-expression $e$ such that $c:=e$ in $M$.

The declarations in theories and morphisms are subject to typing conditions that are not essential for our purposes here.
We only mention the main theorem that guarantees that well-typed morphisms preserve judgments, i.e., if $\vdash_S e:E$, then $\vdash_T M(e):M(E)$.
This includes the preservation of truth via the propositions-as-types principle where $E$ is a proposition and $e$ its proof.

An \mmt \textbf{diagram} consists of a set of named structured module declarations.
\begin{nomodexp}
For a given diagram, we write $\Thy$ for the set of theory names.
And we write $\Mo{S}{T}$ for the set of morphisms defined by
\begin{compactitem}
 \item for every declaration $m:S\to T=\{\sigma\}$, we have $m\in\Mo{S}{T}$,
 \item for every $T\in\Thy$, we have $\id{T}\in\Mo{T}{T}$,
 \item for every $M\in\Mo{R}{S}$ and $N\in\Mo{S}{T}$, we have $M;N\in\Mo{R}{T}$.
\end{compactitem}
$\Thy$ and $\Mo{S}{T}$ form the category of theories and morphisms.
\end{nomodexp}
\begin{modexp}
Every diagram induces sets of anonymous module expressions.
These are mutually recursive: the names of the former provide the base cases for inductively forming the latter, and the latter may occur in the bodies of the former.
Every diagram induces sets
\begin{compactitem}
 \item $\thy$ of identifiers of theories $t$,
 \item $\Thy\supseteq \thy$ of theory expressions $T$,
 \item $\mo{S}{T}$ of identifiers of atomic morphisms $m$ from $S$ to $T$, and
 \item $\Mo{S}{T}\supseteq\mo{S}{T}$ of morphism expressions $M$ from $S$ to $T$.
\end{compactitem}
\end{modexp}
Then flattening assigns
\begin{compactitem}
 \item to each $T\in\Thy$ the flat theory $\flt{T}$,
 \item to each $M\in\Mo{S}{T}$ the flat morphism $\flt{M}$ from $\flt{S}$ to $\flt{T}$.
\end{compactitem}

\paragraph{Well-Formed Expressions}
The inference system to define well-formed expressions is not a primary interest of this paper.
Intuitively, the set $\Exp{T}$ of expressions over $T$ includes all syntax trees that can be formed using the constants from $\flt{T}$.
Moreover, well-formed modules may only use expressions that satisfy certain typing and equality judgments.
We refer to \cite{rabe:howto:14} for details.

The only aspect of the inference system for these judgments that is relevant to our purposes here is the rules for constants:
\[\rul{c:E[=e] \minn \flt{T}}{\vdash_T c[=e]:E}\]
Here, in order to avoid case distinctions for the case where a definition is present or not, we use a combined typing+equality judgment: $e_1=e_2:E$ represents the conjunction of $e_1:E$, $e_2:E$, and $e_1=e_2$.
Correspondingly, the definition of the homomorphic extension of a morphism with domain $S$ includes the following case for constants:
\[M(c) = \cas{e\mifc (c:E) \in \flt{S}, (c:=e)\in\flt{M} \\ M(e) \mifc (c:E=e)\in \flt{S}}\]
Here if $c$ has a definiens in $S$, we expand it before applying $M$.%
\footnote{The \mmt tool also $c:e\in \flt{M}$ even if $(c:E=e')\in\flt{S}$ has a definition. In that case, \mmt checks $\vdash_T M(e')=e$ and puts $M(c)=e$. This is important for efficiency but not essential for our purposes here.}

Note that these base cases introduce a mutual recursion between well-formedness and flattening:
the well-formedness of a declaration in a theory depends on the flattening of all preceding declarations;
correspondingly, the well-formedness of an assignment in a morphism depends on the homomorphic extension of the morphism obtained by flattening of all preceding assignments.
Vice versa, well-formedness is a precondition for defining the flattening --- the definition of flattening may become nonsensical if applied to ill-formed modules.

It is desirable to define well-formedness independently of flattening.
But the mutual recursion makes sense from an implementation perspective: typical tools for formal systems first parse, check, and flatten a declaration entirely before moving on to the next declaration.
Moreover, often the checking of well-formedness is as difficult or expensive as flattening anyway.
In fact, inspecting practical systems with modular features (which often do not have a full rigorous formal specification) shows that our definition elegantly captures the essential commonalities between them.

Therefore, we will make flattening a partial function, i.e., $\flt{X}$ is undefined if the module $X$ is not well-formed. 

%Consider any declaration $n[:E][=e]$ in the body of a theory $t$.
%It is well-formed if $n$ is a fresh name $E$ and $e$ are well-formed expressions over $t$.
%(Additionally, $E$ and $e$ must satisfy typing conditions that are not essential for our purposes.)
%Importantly, $E$ and $e$ must be from $\Exp{\flt{t}}$ using only constants that were introduced in preceding declarations.
%
%Similarly, consider an assignment $c:=e$ in a morphism $m$ from $S$ to $T$, where $c:E$ is a declaration is $S$.
%It is well-formed if $\vdash_T e:\flt{m}(E)$.

\paragraph{Logics in \mmt}
\ednote{explain that we use a fixed instance of MMT which adds LF and FOL; only relevant for examples}

\subsection{Syntax}

We start with the syntax for theories (which arises as a special case of the one given in \cite{RK:mmt:10}):

\begin{definition}[Theory]\label{def:theory}
The grammar for theories and expressions is of the form
\begin{grammar}
TDec     & T=\{Dec,\ldots,Dec\}  & \text{theory declaration} \\
Dec      & n:E\opt{=E}           & \text{constant declaration}\\
c        & t?n                   & \text{qualified constant identifiers} \\
E        & c \alt \ldots         & \text{expressions built from constants} \\
\end{grammar}

In a theory declaration, each symbol \textbf{name} $n$ may be declared only once, and its \textbf{type} and \textbf{definiens} (if present) must be closed expressions only previously introduced constants.
We omit the remaining productions for expressions here, which allow forming complex expressions using application, binding, variables, literals, etc.
\end{definition}

\begin{example}\label{ex:thgroup}
The (flat) theories $\cn{Group}$ and $\cn{DivGroup}$ from the introduction are:
\begin{mmtcode}
Group =
  U	: type
  op		: U ⟶ U ⟶ U	 # 1 ∘ 2 
  unit		: U
  inverse : U ⟶ U   # 1 ⁻¹ 
  // axioms omitted
\end{mmtcode}
\begin{mmtcode}
DivisionGroup =
  U : type
  div : U ⟶ U ⟶ U  # 1 / 2 
  unit : U
  // axioms omitted
\end{mmtcode}
Here we use \lstinline|# <not>| to indicate notations; these are part of the language supported by \mmt but omitted from the formal grammar above because we only need them here to present legible examples.

Moreover, we use a few constants from a type systems\ednote{finish}$\forall,\doteq$ etc.
\end{example}

At this point, this grammar only allows forming flat theories, i.e., $\Thy=\thy$.
It is intentionally independent of the concrete choice of modularity principles, i.e., the kinds of module-structuring declarations (e.g., includes) and the module-expression-forming operators (e.g., union).
But it already introduces all non-terminals for modularity: new declaration kinds for structured theories arise by adding production for the non-terminal $Dec$ and new theory expressions are formed by adding productions for the non-terminal $T$.
This makes it easy to incrementally add new declaration kinds and operators to the language.

%These theories are similar to OpenMath content dictionaries \cite{openmath}.
%In particular, \mmt uses OpenMath-style qualified symbol identifiers $t?n$.%
%\footnote{Of course, the \mmt IDE and web interface parse and display these identifiers only as $n$ (if that is unambiguous), especially if they are used in formulas.
%But the interface always knows the qualified identifier and shows it, e.g., on hover.}
%Thus, symbols of the same name declared in different atomic theories are always distinguished.

We proceed accordingly for flat morphisms:

\begin{definition}[Morphism]\label{def:morphism}
A \textbf{morphism} is of the form
\begin{grammar}
MDec   & m:T\to T=\{Ass,\ldots,Ass\}     & \text{flat morphism declaration}\\
Ass    & c:=o                            & \text{assignment to symbol}\\
M      & m \alt \id{T} \alt M;M          & \text{morphism expressions}
\end{grammar}
such that a morphism declaration contains exactly one assignment $c:=o$ for each $c\in\dom{S}$, all of which satisfying $o\in\Exp{T}$.
\end{definition}

\begin{example}[Morphisms]\label{ex:dg2g}
We give the morphism \cn{DG2G} between the theories from Ex.~\ref{syn:incl}:
\begin{mmtcode}
DG2G : DivisionGroup -> Group =
  U := U
  div := [a,b] a ∘ (b⁻¹) 
  unit := unit
  // assignments to axioms omitted
\end{mmtcode}
It maps the universe and unit of a division group to the corresponding notions of a group.
And we have $\cn{DG2G}(a/b)=a\circ b^{-1}$.
Additionally, the morphism must maps every axiom of \cn{DivGroup} to a proof in \cn{Group} of the translated statement, but we omit those assignments.
\end{example}

%Crucially, morphisms \textbf{preserve judgments}, in particular they map theorems to theorems.
%Formally, an atomic morphism $m:S\to T$ is \emph{well-formed} if
%\begin{compactitem}
%\item whenever $c$ has type $t$ in $\flt{S}$ and $c:= o$ in $\flt{m}$, then $\der_T o:m(t)$,
%\item whenever $c$ has definiens $d$ in $\flt{S}$ and $c:= o$ in $\flt{m}$, then $\der_T o\equiv m(d)$.
%\end{compactitem}
%Then we can show that for every $M\in\Mo{S}{T}$ if $\der_S o:o'$ (or $\der_S o\equiv o'$), then $\der_T M(o):M(o')$ (or $\der_T M(o)\equiv M(o')$).
%The details can be found in \cite{rabe:howto:14}.
%%In particular, this means that $M$ maps $S$-theorems to $T$-theorems.

Finally, we define diagrams as collections of modules:

\begin{definition}[Diagram]
A \textbf{diagram} is a list of theory and morphism declarations:
\begin{grammar}
Dia    & \rep{(TDec \alt MDec)}          & \text{diagrams}\\
\end{grammar}
A diagram is well-formed if each atomic theory/morphism declaration has a unique name is well-formed relative to the diagram preceding it.
\end{definition}

We give some examples how to add modularity principles to the grammar:

\begin{example}[Includes]\label{syn:incl}
We extend the grammar with
\begin{grammar}
  Dec & \icl{S} & \text{include a theory into a theory} \\
  Ass & \icl{M} & \text{include a morphism into a morphism}
\end{grammar}
This allows writing the theory $\cn{Group}$ from Ex.~\ref{ex:thgroup} by extending a theory of monoids.
Again omitting all axioms, this looks as follows:
\begin{mmtcode}
Monoid =
  U: type
  op : U ⟶ U ⟶ U	 # 1 ∘ 2 
  unit	: U
Group =
  include Monoid
  inverse : U ⟶ U   # 1 ⁻¹ 
\end{mmtcode}
\end{example}

\begin{union}
\begin{example}[Union]\label{syn:union}
We extend the grammar with
\begin{grammar}
  T   & T\cup T & \text{union of theories} \\
  M   & M\cup M & \text{union of morphisms}
\end{grammar}
\end{example}
\end{union}

\begin{modexp}
\begin{example}[Category of Theories]\label{syn:cat}
To obtain the category structure of theories and morphisms, we need identity and composition.
We add them to the grammar as follows:
\begin{grammar}
  M   & \id{T} & \text{identity morphism} \\
  M   & M;M    & \text{morphism composition}
\end{grammar}
\end{example}
\end{modexp}

For each example, we will define the semantics in the next section.

\subsection{Semantics}

\paragraph{Flattening}
Just like the syntax, the definition of flattening is \textbf{compositional} in the sense that new modularity principles can be added later independently of each other.
In particular, for every module expression $X$, we define its flattening $\flt{X}$ by induction on the syntax.
This follows the principle of compositional semantics, e.g., $\flt{f(X,Y)}$ depends only on $\flt{X}$ and $\flt{Y}$. 

For theories $T$, at this point, this induction only has the base case of a reference to a declared theory $t$.
Further cases arise when we add structuring declarations.

For the base case, of a theory $t=\{\Sigma\}$, we define $\flt{t}$ by induction on the declarations in $\Sigma$:
\[\flt{t}=\flt{\Sigma}\]
where
 \[\flt{\cdot}=\es\]
 \[\flt{(\Sigma,D)}=\flt{\Sigma}\cup\fltd{\flt{\Sigma}}{D}\]
Here $\fltd{T}{D}$ is the flattening of declaration $d$ in the context of theory $T$.

At this point, we only have one case for declarations $D$, namely constant declarations.
Their flattening is trivial:
 \[\fltd{\Sigma}{(n:E[=e])}=\{t?n:E[=e]\}\]
where $t$ is the name of the containing theory.

Correspondingly, for a declared morphism $m:S\to T=\{\sigma\}$, we have $m\in\Mo{S}{T}$ and $\flt{m}$ is defined by induction on the assignments in $\sigma$:
 \[\flt{\cdot}=\es\]
 \[\flt{(\sigma,\;A)}=\flt{\sigma}\cup\fltd{\flt{\sigma}}{A}\]
with the trivial base case
 \[\fltd{\sigma}{(c:=e)}=\{c:=e\}\]
\begin{nomodexp}
Moreover, for $T\in\Thy$ we define
\[\flt{\id{T}}=\{c:=c \;|\; c:E\in\flt{T}\}\]
And for $M;N\in\Mo{R}{T}$, we define
\[\flt{(M;N)}=\{c:=\flt{N}(\flt{M}(c)) \;|\; c:E\in\flt{R}\}\]
Note that in both cases, we only have to consider $R$-constants without definiens.
\end{nomodexp}

\begin{example}[Includes (continued from Ex.~\ref{syn:incl})]\label{sem:incl}
Includes add new declarations $D$, so we have to add cases to the definition of $\fltd{\Sigma}{D}$.
We do this as follows
  \[\fltd{\Sigma}{(\icl{S})} = \flt{S}\]
This has the effect of copying over all declarations from the included into the including theory.

Note that the definition of $\fltd{\Sigma}{n:E=e}$ qualifies the constant name $n$ with the theory name $t$ to form the identifier $t?n$ in the flattening.
Therefore, includes can never lead to name clashes.

Also note, that because $\flt{S}$ is a \textit{set} of declarations, the include relation is transitive: if $t$ includes $s$ via two different paths, $\flt{t}$ only contains one copy of the declarations of $\flt{s}$.
The situation is slightly more complicated for morphisms: if a morphism out of $t$ includes two different morphisms out of $s$, these have to agree.
Therefore, flattening and well-formedness must be distinguished:
  \[\fltd{\sigma}{(\icl{M})} = \flt{M}\]
under the condition $\flt{M}$ agrees with $\flt{\sigma}$ on any constant that is in the domain of both.

The flat variant $\cn{Group}^\flat$ of our refactored theory $\cn{Group}$ from \autoref{syn:incl} is hence given by copying all included symbols (from $\cn{Monoid}$) over to $\cn{Group}$, resulting in exactly the theory as presented earlier in \autoref{ex:thgroup} except for global identifiers.
\end{example}

\begin{union}
\begin{example}[Union (continued from Ex.~\ref{syn:union})]\label{sem:union}
We define the semantics of unions of theories as
\[\flt{(S\cup T)}=\flt{S}\cup\flt{T}\]
The handling of name clashes is the same as for includes:
Because the theories declared in a diagram have unique names and declarations in the flattening are qualified by theory names, union theories cannot have name clashes.
But we have to watch out when taking unions of morphisms:
We define $M_1\cup M_2\in \Mo{S_1\cup S_2}{T}$ if $M_i\in\Mo{S_i}{T}$ and $M_1$ and $M_2$ agree for every constant in $\dom{\flt{S_1}}\cap\dom{\flt{S_2}}$.
In that case, we put
\[\flt{(M\cup N)}=\flt{M}\cup\flt{N}\]
\end{example}
\end{union}

%Includes between theories are always well-formed (assuming the included theory is).
%But includes between morphisms are more difficult -- we have to impose two restrictions.
%Firstly, an assignment $\icl{M}$ is well-formed in an atomic morphism $m:S\to T$ if $R\harr S$ and $M\in\Mo{R}{T}$.
%Its intended effect is that $m$ maps all $c\in\dom{R}$ to $M(c)$.
%That yields the second restriction: For any two morphisms $M_i\in\Mo{R_i}{T}$ included into $m$ and any theory $C$ satisfying $C\harr R_1$ and $C\harr R_2$, we require that $M_1|_C\equiv M_2|_C$.
%Thus, any two included morphisms must agree on their common domain $C$; that is necessary so that $\flt{m}$ does not contain multiple assignments for the same identifier.

\begin{modexp}
\subsection{Semantic Properties}

For two theories $S,T\in\Thy$, we say that $S$ is \textbf{included} into $T$, written $S\harr T$, if $\flt{S}\sq\flt{T}$.
%Clearly, $m(o)=o$ for all $S$-objects $o$.
Thus, every declaration of $S$ is also available in $T$ and $\Exp{S}\sq\Exp{T}$.
This induces an inclusion morphism, which maps every $c\in\dom{S}$ to itself.
We do not introduce a name for that morphism and simply define $\id{S}\in\Mo{S}{T}$ whenever $S\harr T$ (which includes the special case $S=T$).
Moreover, if $S\harr T$ and $M\in\Mo{T}{U}$, we define the restriction $M|_S\in\Mo{S}{U}$ of $M$ to $S$ by $\id{S};M$.

We say that two theories $S,T\in\Thy$ are \textbf{equal}, written $S\equiv T$, if $\flt{S}=\flt{T}$.
This is equivalent to $S\harr T$ and $T\harr S$.

Correspondingly, we say that two morphisms $M,N\in\Mo{S}{T}$ are \textbf{equal}, written $M\equiv N$, if $\flt{M}=\flt{N}$.
In that case, $M(e)=N(e)$ for all expressions $e\in\Exp{S}$.

Clearly, $\equiv$ is an equivalence relation and $\harr$ is an order relation (if anti-symmetry is taken with respect to $\equiv$).

\begin{example}[Includes (continued from Ex.~\ref{sem:incl})]\label{rel:incl}
We have the following soundness theorem for includes: if a theory is declared as $t=\{\Sigma\}$ and $\Sigma$ contains $\icl{S}$, then $S\harr t$.

Note that in the absence of include declarations or unions, $s\harr t$ never holds because all $c\in\dom{s}$ are of the form $s?n$ and all $c\in\dom{t}$ are of the form $t?n$.
In fact, $\flt{s}$ and $\flt{t}$ unless $s=t$.
\end{example}

\begin{union}
\begin{example}[Union (continued from Ex.~\ref{sem:union})]\label{rel:union}
We have the following soundness theorem for unions: $S_i\harr S_1\cup S_2$.
Moreover, $S_1\cup S_2$ is a colimit in the category of theories and morphism, and $M_1\cup M_2\in\Mo{S_1\cup S_2}{T}$ is the universal morphism out of it.
\end{example}
\end{union}

\begin{example}[Category of Theories (continued from Ex.~\ref{sem:cat})]\label{rel:cat}
Identity and composition are congruent with respect to $\equiv$.
Moreover, composition is associative and identity neutral with respect to $\equiv$.
Thus, each diagram indeed yields the structure of a category with objects $\Thy$ and arrows $\Mo{S}{T}$.
\end{example}
\end{modexp}

%\section{Inclusion Morphisms}\label{sec:include}
%\subsection{Motivation}

\mmt as introduced in \cite{RK:mmt:10} allowed only two declarations in atomic theories: symbol declarations $c$ and structure declarations, which instantiate and import a theory.
(We omitted the latter in Sect.~\ref{sec:mmt}.)
We will now introduce include declarations to \mmt.

We can see include and structure declarations as duals.
Both import a previously declared theory, and both are transitive in the sense that if $T$ imports $S$, it also imports anything $S$ imports.
They differ in the sharing semantics when the same theory is imported multiple times: Multiple includes of $S$ are redundant and introduce a single copy of $S$; multiple structures of $S$ introduce multiple fresh copies of $S$.

Consequently, include declarations are \emph{unnamed} and the importing theory refers to the included symbols using the identifiers from $\dom{S}$.
Structure declarations on the other hand introduce a name in order to form qualified names for the fresh copy of the imported symbols.

\subsection{Definition}

We extend the \mmt grammar with
\begin{grammar}
  Dec & \icl{S} & \text{include a theory into an atomic theory} \\
  Ass & \icl{M} & \text{include a morphism into an atomic morphism}
\end{grammar}

We extend the definitions of $\flt{\Sigma}$ with the case
  \[\flt{(\Sigma,\;\icl{S})} = \flt{\Sigma}\cup\flt{S}\]
and the definition of $\flt{\sigma}$ with the case
  \[\flt{(\sigma,\;\icl{M})}=\flt{\sigma}\cup\flt{M}\]
%  \[\flt{(\Sigma,\;\strc{s}{S})} = \flt{\Sigma}\cup\{t?s/n[:t'][=d'] | n[:t][=d]\in \flt{S}\}\]
%where $t'$ and $d'$ arise from $t$ and $d$ by renaming all symbol identifiers accordingly.
%  \[\flt{(\sigma,\;\strca{M})}=\flt{\sigma}\cup\{t?s/n:=M(\flt{M}\]
In particular, this yields $S\harr t$ whenever $t$ declares an include of $S$.

Includes between theories are always well-formed (assuming the included theory is).
But includes between morphisms are more difficult -- we have to impose two restrictions.
Firstly, an assignment $\icl{M}$ is well-formed in an atomic morphism $m:S\to T$ if $R\harr S$ and $M\in\Mo{R}{T}$.
Its intended effect is that $m$ maps all $c\in\dom{R}$ to $M(c)$.
That yields the second restriction: For any two morphisms $M_i\in\Mo{R_i}{T}$ included into $m$ and any theory $C$ satisfying $C\harr R_1$ and $C\harr R_2$, we require that $M_1|_C\equiv M_2|_C$.
Thus, any two included morphisms must agree on their common domain $C$; that is necessary so that $\flt{m}$ does not contain multiple assignments for the same identifier.

\subsection{Discussion}

The original structure declarations of \mmt are significantly more expressive than include declarations: They allow fine-tuning the sharing by instantiating some symbols of the imported theory.
In particular, they can emulate include declarations as a special case.

But the inclusion relation $S\harr T$ induced by include declarations has a number of practically valuable properties.
Therefore, the author chose to add include declarations even though they do not adding expressive strength.
Indeed, the practical advantages are huge.
For example, after implementing support for inclusions, the author refactored the LATIN atlas \cite{CHKMR:latinabs:11} -- a large case study on formalizing logics using an \mmt style module system.
At this point it contains approximately $2000$ imports, of which $1600$ were changed to inclusions.
\medskip

There are several reasons for these advantages.
Firstly, inclusion is an order relation between theories: Each theory can be included at most once into another one.
That makes the inclusion relation very easy and efficient to maintain.

Secondly, we can form canonical qualified identifiers $s?n$ for identifiers of $s$ included into a theory $t$.
These are canonical identifiers that uniquely identify the included symbols and that are obvious to users.
Users can easily see where a symbol was imported from, but the import chain along which it was imported remains transparent.
%A qualified identifier $q.c$ where $q$ is provided by the import.
%This quickly leads to long names like $p.q.r.c$. Moreover, because the import path is part of the name, refactoring becomes more costly and multiple qualified identifiers may actually be the same.

Thirdly, $\Exp{s}\sq\Exp{t}$ so that the included declarations do not have to be translated from $s$ to $t$.
Consequently, implementations do not have to duplicate the translations: They can store all local declarations with the declaring theory, and use the relation $s\harr t$ to determine which theory may access which declarations.
Tools that duplicate imported declarations on the other hand quickly suffer from exponential blowups when handling large libraries.

Another advantage is that inclusion allows a form of implicit conversion.
As explained in \cite{RK:mmt:10}, we can consider morphisms $M\in\Mo{T}{F}$ as models of $T$ defined in terms of $F$.
For example, consider $T=\cn{Group}$ and $F=\cn{ZFC}$; then concrete group corresponds to morphisms $G\in\Mo{T}{F}$.
If we further assume $\cn{Monoid}\harr \cn{Group}$, then $G|_{\cn{Monoid}}$ is $G$ seen as a monoid.
Moreover, we can easily make the conversion $G\mapsto G|_{\cn{Monoid}}$ implicit without confusing humans or tools.
\medskip

Of course the above advantages are not a new discovery.
Many languages have long used a combination of unnamed and named imports (e.g., inheritance and aggregation in object-oriented programming correspond to includes and structures).
However, the above appears to be the first time that the semantics of include declarations is defined formally in the context of OpenMath-style qualified identifiers.

%\begin{compactitem}
%\item The original identifier $c$. This quickly leads to name clashes.
%\end{compactitem}
%There is one special case, where choosing identifies is easy: If $S\harr T$, e.g., $S$ is included into $T$.
%Then we can use $S.c$ as a qualified identifier.
%
%The reason why $S.c$ works as a qualified identifier is that $S$ can only be included into $T$ once and in a uniquely determined way.
%Implicit morphisms generalize this property: $S$ can only be implicitly mapped into $T$ once and in a that is uniquely determined by the implicit-diagram.
%Consequently, we can use any declaration $c$ of $S$ in $T$ using the identifier $S.c$.


\section{Implicit Morphisms}\label{sec:impl}
\subsection{Motivation}

The motivation behind implicit morphisms is to generalize the benefits of include declarations even further.
Sometimes an atomic morphism is conceptually similar to an inclusion: For example, we might want to declare $m:\cn{Division}\to\cn{Group}=\{\cn{divide} := \lambda x,y.x\circ y^{-1}, \;\ldots\}$ to show how division is ``included'' into the theory of groups.
Other times we have to use structures because we need to duplicate declarations but we would still like to use them like includes.
For example, we have to declare the theory of rings using two structures (for the commutative group and the monoid).
But we still want a ring to be implicitly converted into, e.g., a commutative group.

Our key idea is to use a commutative{\footnotemark} subdiagram of the \mmt diagram, which we call the \emph{implicit-diagram}.
It contains all theories but only some of the morphisms -- the ones designated as \emph{implicit}.
Because the implicit-diagram commutes, there can be at most one implicit morphism from $S$ to $T$ -- if this morphism is $M$, we write $\ipc{M}{S}{T}$.
\footnotetext{A diagram commutes if for any two morphisms $M,M'\in\Mo{S}{T}$, we have $M\equiv M'$.}
%Practically, this hasis a multi-graph whose nodes are the theories and whose edges are the atomic morphisms.
%We can extend the correspondence by identifying all paths with the composition of all edges along the path.
%Then concatenation of paths is just composition of morphisms, and the empty path from $T$ to itself is the identity morphism $\id{T}$.
%(This is well-defined because composition is associative and identity is neutral.)
%A diagram is commutative if any two paths between the same nodes are equal (with respect to $\equiv$), i.e., there is at most one morphism between any two nodes.

The implicit-diagram generalizes the inclusion relation $\harr$.
In particular, all identity and inclusion morphisms are implicit, and we recover $S\harr T$ as the special case $\ipc{\id{S}}{S}{T}$.
Moreover, the relation ``exists $M$ such that $\ipc{M}{S}{T}$'' is also an order.

Consequently, many of the advantages of inclusions carry over to implicit morphisms.
\begin{compactitem}
\item It is very easy to maintain the implicit-diagram, e.g., as a partial map that assigns to a pair of theories the implicit morphism between them (if any).
\item We can generalize the visibility of identifiers: If $\ipc{M}{S}{T}$, we can use all $S$-identifiers in $T$ as if $S$ were included into $T$.
Any $c\in\dom{S}$ is treated as a valid $T$-identifiers with definiens $M(c)$.
\item We can still use canonical identifiers $s?n$. Because there can be at most one implicit morphism $\ipc{M}{s}{T}$, using $s?n$ as an identifier in $T$ is unambiguous.
Crucially, we can use $s?n$ without bothering what $M$ is.
\end{compactitem}

%Thus the main question arises what other morphisms we should make implicit.

\subsection{Definition}

\paragraph{Syntax}
We have already indicated that we will consider all identities and inclusions as implicit.
It remains to give users the option introduce additional implicit morphisms.
For that purpose, we allow two \mmt declarations to carry an attribute: 
\begin{grammar}
MDec   & Att\; m:T\to T=\{Ass,\ldots,Ass\}   & \text{attributed atomic morphism} \\
Dec    & Att\; \icl{S}                       & \text{attributed inclusion}\\
Att    & \keyword{implicit} \alt \keyword{invertible} & \text{attributes}
\end{grammar}

Every morphism with the attribute \keyword{implicit} occurs in the implicit-diagram.
(Of course, this is redundant for includes.)
Every morphism with the attribute \keyword{invertible} is also considered implicit, and additionally so is its inverse morphism.

We do not even have to provide a name for the inverse morphism: It is uniquely determined (up to $\equiv$) anyway, and it will be used only implicitly.
We only have to make sure that the inverse actually exists and that it can be found automatically.
This is not decidable in general, but it is decidable in many important special cases, in particular, renamings and definitional extensions.

\paragraph{Semantics}
Well-formedness and semantics are defined in parallel.

\begin{definition}[Implicit-Diagram]
Given a diagram $D$, its implicit-diagram $D^i$ contains all theories and the following morphisms:
\begin{compactitem}
\item all identities and inclusions,
\item all compositions of implicit morphisms,
\item every morphism with the \keyword{implicit} attribute,
\item for each morphism with the \keyword{invertible} attribute, that morphism and its inverse. 
\end{compactitem}
\end{definition} 

The only requirement for well-formedness is that the implicit-diagram always commutes.
However, in general, commutativity may be undecidable.
Similarly, the existence of an inverse morphism may be undecidable.
Therefore, we use sound criteria for commutativity and invertibility as defined in Sect.~\ref{sec:commute} and \ref{sec:inverse}.

\begin{definition}[Well-Formedness]
Consider a well-formed diagram $D, m:S\to T=\{\sigma\}$.
Let $D^a$ be $D, a\; m:S\to T=\{\sigma\}$.
$D^{\keyword{implicit}}$ is well-formed if $D^i$ extended with an edge for $m$ commutes.
$D^{\keyword{invertible}}$ is well-formed if $D^i$ extended with edges for $m$ and its inverse commutes.

Correspondingly, consider a well-formed diagram $D$ of the form $D_0, T=\{\ldots,\icl{S},\ldots\}$.
Let $D^a$ be $D_0, T=\{\ldots, a\; \icl{S},\ldots\}$.
$D^{\keyword{invertible}}$ is well-formed if the inclusion has an *inverse.
\end{definition}

Finally, we modify the definition of $\Exp{T}$ and $M(-)$ to add all implicitly imported identifiers:

\begin{definition}[Visible Declarations]
The set $\flti{T}$ contains the following declarations:
for every $\ipc{M}{S}{T}$ and every $c[:t][=d]$ in $\flt{S}$, the declaration $c[:M(t)]=M(c)$.

The set $\Expi{T}$ contains all objects formed from the symbols in $\flti{T}$.
The map $M(-)$ maps every $c$ for which $\flt{M}$ does not provide an assignment to $M(d)$ where $d$ is the definiens of $c$.
\end{definition}

This modification is conservative in the sense that $\flt{T}\sq\flti{T}$ and $\flti{T}\sm\flt{T}$ contains only declarations with definiens.
Moreover, $\Exp{T}\sq\Expi{T}$ and $M(-)$ remains unchanged on all $o\in\Exp{T}$.
It is easy to see that $M(-)$ still preserve all judgments.

\subsection{Commutativity}\label{sec:commute}

To prove commutativity, we have to find all paths in the implicit-diagram and check $M\equiv M'$ for all pairs of morphisms between the same nodes.

\mmt reduces $M\equiv M'$ for $M,M'\in\Mo{S}{T}$ to a first-order problem in the theory of categories, which uses the following axioms:
\begin{compactitem}
\item associativity of composition,
\item neutrality of identity,
\item $M;M'\equiv \id{S}$ as well as $M';M\equiv\id{T}$ if $M\in\Mo{S}{T}$ and $M'$ is the known *inverse of $M$,
\item if an atomic morphism $m$ includes $M\in\Mo{R}{T}$, then $M|_R\equiv M$.
\end{compactitem}

This is a sound but not complete axiomatization of $M\equiv M'$.
We can obtain a complete axiomatization in a totally different way: Check $\der_T M(c)\equiv M'(c)$ for all $c\in\dom{S}$.
However, this can be extremely expensive: It requires computing $\dom{S}$ and proving an object level equality for each identifier.
Even if the equality of objects is decidable, this is often too expensive.

In fact, because implicit morphisms are meant to be used in very particular cases, we expect the above incomplete calculus to perform reasonably well in practice.
If showing the equality of certain morphisms is hard, one should probably not make them implicit in the first place.


\subsection{Inverse Morphisms}\label{sec:inverse}

We only two use sufficient criteria to determine an *inverse of a morphism $M$.

\paragraph{Conservative Extensions}
Intuitively, $t$ extends $S$ definitionally, if it only adds defined symbols, e.g., abbreviations or theorems.
That is the most important case of conservative extensions.

\begin{definition}[Definitional Extension]
An include declaration \[t=\{\keyword{invertible}\;\icl{S},\Sigma\}\] is \textbf{definitional} if
\begin{compactitem}
 \item all symbol declarations in $\Sigma$ have a definiens,
 \item all include declarations $\icl{R}$ in $\Sigma$ are such that $R$ also definitionally includes $S$.
\end{compactitem}
\end{definition}

There are multiple, subtly different definitions of conservative extension.
However, definitional extensions as defined above are always conservative (i.e., for any reasonable definition of conservativity and in any reasonable formal language).
In fact, they are always invertible: The inverse morphism $i:t\to S$ maps
\begin{compactitem}
 \item an identifier $c\in\dom{S}$ to $c$ and everythemselves and the others to their definiens,
 \item any other identifier, which must have a definiens $d$, to $i(d)$.
\end{compactitem}

%We use two auxiliary definitions.
%Given $u:T\to U$, we write $u|_S$ for the restriction of $u$ to $S$, i.e., the composition of the inclusion $S\harr T$ and $u$.
%An extension $S\to T$ is retractable if there is a morphism $r:T\to S$ such that $r|_S=\id{S}$.
%
%Whether a given extension $S\harr T$ is conservative depends on the specific definition of conservativity and the used logic.
%However, the most important kind of conservative extension is very easy to handle: the extension of a theory with a defined constant (e.g., an abbreviation or a theorem).
%These extensions are always conservative (i.e., for any reasonable definition of conservativity and in any reasonable logic).
%In fact, they are always isomorphic to the original theory.
%
%Extensions with definition are almost exclusively used to add something to $S$.
%Therefore, it is reasonable to automatically designate the retraction as implicit.
%In particular, this provides an elegant way to add theorems to $S$ in such a way that they are available to any theory importing $S$, but without actually changing $S$.
%
%More generally, any retractable extension $S\harr T$ is always conservative.
%But if the retraction usually does not exist uniquely (Often an arbitrary non-canonical choice is needed, e.g., any type.), this is not enough to make the retraction implicit.

\paragraph{Renamings}
Intuitively, an atomic morphism $m:S\to T$ is a renaming if it maps all $S$-identifiers to $T$-identifiers (rather than $T$-objects).

\begin{definition}[Renaming]
Consider a morphism $M\in\Mo{S}{T}$.
Let $A\sq\dom{S}$ and $B\sq\dom{T}$ contain those identifiers without definiens.
$M$ is an (injective, surjective) \textbf{renaming} if the restriction of $M(-)$ is an (injective, surjective) mapping $A\to B$.
\end{definition}

Clearly, if $m$ is an injective and surjective renaming, it is straightforward to obtain its inverse by inverting $M(-)$.

\subsection{Applications}

Canonical isomorphisms.

User-declared atomic morphisms.

\paragraph{Extending Theories from the Outside}

\paragraph{Transparent Refactoring}
A major drawback of OpenMath/\mmt-style qualified identifiers is that it can preclude transparent refactoring.
For example, consider a theory $t=\{\icl{r},\icl{s}\}$ and assume we want to move a symbol $n$ from $r$ to $s$.
This requires changing any qualified reference to $r?n$ (in any theory including $t$) to $s?n$.
That can be very costly, especially in large libraries where the refactorer might not even know of or have write access to all theories importing $t$.

With implicit morphisms, we rename $s$ to $s'$, move $n$ from $s'$ to $r$, and recover $s$ by $s=\{\icl{s'},n\}$.
Now we can change define $t=\{\icl{r},\icl{s'}\}$ and give an implicit morphism from $s$ to $t$, which maps $s?n$ to $r?n$.
All existing references to $s?n$ stay well-formed (and serve as aliases for $r?n$) so that no further changes in theories importing $t$ are needed.


\section{Applications}\label{sec:appl}
\subsection{Identifying Theories via Implicit Isomorphisms}\label{sec:inverse}

In this section, we introduce several language extensions that introduce implicit isomorphisms.

Note that because identity morphisms are implicit, our uniqueness requirement for implicit morphisms implies that two theories $S$ and $T$ must be isomorphic if there are implicit morphisms in both directions.
Moreover, making a pair of isomorphisms implicit is well-formed if there are no other implicit morphisms between $S$ and $T$ yet.

\paragraph{Renamings}
We say that a named morphism $r:S\to T=\{\ldots\}$ is a \textbf{renaming} if
\begin{compactitem}
 \item all assignments in its body are of the form $c:=c'$ for $T$-constants $c'$ without definiens
 \item every such $T$-constants $c'$ occurs in exactly one assignment.
\end{compactitem}
Clearly, every renaming is an isomorphism.
The inverse morphisms contains the flipped assignments $c':=c$.

We make the following extension to syntax and semantics:
\begin{compactitem}
  \item A morphism declaration $r:s\to t=\{\ldots\}$ may carry the attribute \keyword{renaming}.
  \item This is well-formed if there are no implicit morphism between $s$ and $t$ yet.
  \item In that case, we define $r\in\IMo{s}{t}$ and $r^{-1}\in\Mo{t}{s}$.
\end{compactitem}

\ednote{@DM: add example, e.g., a definition of Monoid that uses different names}

\paragraph{Definitional Extensions}
We say that the named theory $t$ is a \textbf{definitional extension} of $S$ if $t=S$ or the body of $t$ contains
\begin{compactitem}
 \item only constant declarations with definiens,
 \item only include declarations of theories that are definitional extensions of $S$.
\end{compactitem}
\ednote{@DM: give example that extends Group with the theorem $(x^{-1})^{-1}\doteq x$ (with omitted proof)}

If $t$ is a definitional extension of $S$, it is easy to prove that $t$ and $S$ are isomorphic: both isomorphisms map all constants without definiens to themselves. In particular, the isomorphism $S\to t$ maps $S$-constants to themselves and expands the definiens of all other constants.

We make the following extension to syntax and semantics:
\begin{compactitem}
  \item An include declaration $\icl{s}$ of a named theory $s$ inside theory $t$ may carry the attribute \keyword{definitional}.
  \item In that case, we define $\id{s}\in\IMo{t}{s}$ (in addition to the implicit morphism $\id{t}\in\IMo{s}{t}$ which is induced by the inclusion).
\end{compactitem}

\ednote{@DM: repeat previous example now with \keyword{definitional} and give another definitional extensions of Group; this new extension may already use the implicitly available definition of the first extension}

\begin{remark}[Conservative Extensions]
A definitional extension is a special case of a conservative extension.
More generally, all retractable extensions are conservative, i.e., all extensions $S\harr T$ such that there is a morphism $r:T\to S$ such that $r$ is the identity on $S$.

But we cannot make the retractions implicit morphisms in general because they are not necessarily isomorphisms.
\end{remark}

\paragraph{Canonical Isomorphisms}
If we have isomorphisms $m:s\to t$ and $n:t\to s$, we simply spell them out in morphism declarations and add the keyword \keyword{implicit} to both.
This requires no language extensions.

\ednote{@DM: add the isomorphism of DG2G}
\begin{example}\label{group:iso}
We give the morphism $\cn{G2DG}$.

\end{example}

While the first one of these declarations is straightforward, the second one requires checking that $m$ and $n$ are actually isomorphism.
Otherwise, the uniqueness condition would be violated.
Thus, we have to check $m;n=\id{s}$ and $n;m=\id{t}$.
In general, the equality of two morphisms $f,g:A\to B$ is equivalent to $\vdash_B f(c)=g(c)$ for all $c:E\in\flt{A}$.
Thus, if equality of expressions is decidable in the logic that \mmt is instantiated with, then \mmt can check this directly.

However, this does not work in practice.
Already elementary examples require stronger, undecidable equality relations are used:

\begin{example}
Consider the isomorphism from Ex.~\ref{group:iso}.
The result of mapping $x\circ y$ from $\cn{Group}$ to $\cn{DivGroup}$ and back is $x\circ(unit\circ y^{-1})^{-1}$.
Clearly, the group axioms imply that this is equal to $x\circ y$.
But formally that requires working with the undecidable equality of first-order logic.
\end{example}

Therefore, in our running example, we only make one of the two isomorphisms implicit.

In the sequel, we design a general solution to this problem.
It allows systematically proving the equality of two morphisms and using that to make both isomorphisms implicit.
This is novel work, but it requires significant prerequisites and is only peripherally related to implicit morphisms.
Therefore, we only sketch the idea and leave the details to future work.

We add a language feature to \mmt to prove equalities between morphisms:
We add the productions
\begin{grammar}
Dia   & \rep{(TDec \alt MDec\alt MEq)}  &\\
MEq   & \keyword{equal} M=M:T\to T by \{\rep{Ass}\} &\\
\end{grammar}

%Firstly, the intuition of $\keyword{equality} T = \{c:=E,\ldots\}$ is that it provides to every base type $c:\type$ of $T$ a judgment $E:c\to c\to \type$ that defines the $T$-specific equality relation for objects of type $c$.
%Technically, this must be a binary logical relation in the sense of \cite{RS:logrels:12} on $\id{T}$.
%As described in \cite{RS:logrels:12}, this induces an equality predicate $\cn{Equal}_E:E\to E\to \type$ on every type $E$.

We define the declaration $M=N:S\to T by \{\sigma\}$ to be well-formed iff
\begin{compactitem} 
  \item $M:S\to T$ and $N:S\to T$ are well-formed morphisms
  \item $\sigma$ contains exactly one assignment $c:=p$ for every $(c:E)\in\flt{S}$
  \item for each of these assignments $c:=p$, the term $p$ is a proof of $\vdash_T M(c)=N(c)$.
\end{compactitem}

To make $m$ and $n$ from above implicit isomorphisms, we have to do three things: define $m$ and $n$, prove the equalities of $m;n=\id{s}$ and $n;m=\id{t}$, and make $m$ and $n$ implicit.
Note that we cannot make both $m$ and $n$ implicit right away because that is only well-formed after proving the equalities.)
Thus, we define a new attribute \keyword{implicit-later}, which states that a morphism should be considered implicit as soon as subsequent equality proves make it well-formed to do so.

\begin{example}[Isomorphisms]
We can now add declarations
 \[\keyword{implicit}:\cn{DG2G}:\cn{DivGroup}\to\cn{Group}=\text{(as above)}\]
 \[\keyword{implicit-later}:\cn{G2DG}:\cn{Group}\to\cn{DivGroup}=\text{(as above)}\]
 \[\keyword{equal}\cn{G2DG};\cn{DG2G}=\id{\cn{Group}}:\cn{Group}\to\cn{Group}=\text{(omitted)}\]
 \[\keyword{equal}\cn{DG2D};\cn{G2DG}=\id{\cn{DivGroup}}:\cn{DivGroup}\to\cn{DivGroup}=\text{(omitted)}\]
where the isomorphisms are as above and we omit all the equality proofs.
\end{example}

\subsection{Fine-Granular and Flexible Theory Hierarchies}
A common problem when defining modular theory hierarchies is that the most natural include-hierarchy for the most important theories is not necessarily the same as the most comprehensive hierarchy.
For example, Ex.~\ref{syn:incl} defines \cn{Group} with an include from \cn{Monoid}.
Instead, we could have used an intermediate theory and includes $\cn{Monoid}\harr \cn{CancellationMonoid}\harr\cn{Group}$.

It is very common to have increasingly strong theories $R,S,T$, where a design with two includes $R\harr S\harr T$ is not desirable:
\begin{compactitem}
 \item Often $R\harr T$ has been defined first and $S$ only later.
   This is very common because people usually formalize the most important theories (e.g., \cn{Monoid} and \cn{Group}) first.
   But inserting $S$ is not easy in retrospect --- changing the theory hierarchy (which is one of the most fundamental structures of a library) usually presents a very expensive refactoring problem.
   And even if we systematically use includes for every known intermediate theory like $S$ (as done in \cite{mathscheme}), we might later discover a new intermediate theory that should have been added.
 \item Often the most natural axioms to use in $T$ are the same independent of whether $T$ includes $R$ or $S$ (e.g., users might prefer the usual inverse-element axioms in \cn{Group} even if they have included \cn{CancellationMonoid}).
 In that case, the axioms of $S$ become provable in $T$ if we use $R\harr S\harr T$.
 This either causes $T$ to have redundant axioms or requires a more complex include mechanism that allows $T$ to include $S$ in a way that turns some of $S$-axioms into theorems.
\end{compactitem}

Therefore, it is common to use a commuting triangle consisting of two includes $R\harr T$ and $R\harr S$ and one morphism $m:S\to T$.
But this is awkward because the relation ``every $T$ is an $S$'' is now mediated by $m$ rather than being canonical.  
Implicit morphisms provide a simple solution to this problem: we keep the triangle but make the morphism $m$ implicit.
This captures exactly the canonical conversion from $S$ to $T$.

\begin{figure}[htb]
	\begin{center}
		\includegraphics[width=.8\textwidth]{graph.png}
		\caption{Magma hierarchy with includes (gray) and implicit morphisms (black)}\label{fig:magmas}
	\end{center}
\end{figure}

Already the elementary algebraic hierarchy provides countless examples of such situations.
For a small fragment of the hierarchy of magmas, Figure~\ref{fig:magmas} shows one possible design using numerous implicit morphisms.
In particular, it uses some of the examples and features from this paper, e.g., an implicit isomorphism to identify the order-theoretic and the algebraic development of semilattices.

It also uses multiple implicit morphisms to introduce the various intermediate theories between $\cn{Band}\harr\cn{SemiLattice}$. 
All of these are of the form $t=\{\icl{\cn{Band}},\,a: F\}$, e.g., \cn{LeftRegularBand} uses $F=\forall x,z. z\circ x\circ z \doteq z\circ x$.
The implicit morphisms map the constants from \cn{Band} to themselves and the axiom $a$ to a proof.
It is straightforward to prove that this part of the diagram commutes: any two morphisms are identical except for the assignment to the axiom $a$, and these are equal due to proof irrelevance.%
\footnote{Our formalization of bands can be found at \url{https://gl.mathhub.info/MMT/examples/blob/devel/source/bands.mmt}.}
%\footnote{All varieties of bands can be axiomatized in this way.}


%\begin{center}
%\begin{tikzpicture}[xscale=2]
%  \node (s1) at (0,0) {$z\circ x\circ y\circ z\doteq  z\circ x\circ z\circ y\circ z$};
%  
%  \node (s2a) at (-1,-1) {$z\circ x\circ y\doteq  z\circ x\circ z\circ y$};
%  \node (s2b) at (1,-1) {$y\circ x\circ z\doteq  y\circ z\circ x\circ z$};
%  \draw[arrow] (s1) to (s2a) {};
%  \draw[arrow] (s1) to (s2b) {};
%  
%  \node (s3a) at (-2,-2) {$x\circ y\doteq  x\circ y\circ x$};
%  \node (s3b) at (0,-2) {$z\circ x\circ y\circ z\doteq  z\circ y\circ x\circ z$};
%  \node (s3c) at (2,-2) {$x\circ y\doteq  y\circ x\circ y$};
%  \draw[arrow] (s2a) to (s3a) {};
%  \draw[arrow] (s2a) to (s3b) {};
%  \draw[arrow] (s2b) to (s3b) {};
%  \draw[arrow] (s2b) to (s3c) {};
%  
%  \node (s4a) at (-1,-3) {$z\circ x\circ y\doteq  z\circ y\circ x$};
%  \node (s4b) at (1,-3) {$x\circ y\circ z\doteq  y\circ x\circ z$};
%  \draw[arrow] (s3a) to (s4a) {};
%  \draw[arrow] (s3b) to (s4a) {};
%  \draw[arrow] (s3b) to (s4b) {};
%  \draw[arrow] (s3c) to (s4b) {};
%
%  \node (s5) at (0,-4) {$x\circ y\doteq  y\circ x$};
%  \draw[arrow] (s4a) to (s5) {};
%  \draw[arrow] (s4b) to (s5) {};
% 
%  %\node[thy,inner sep=.3cm,fill=gray] (t1) at (0,0) {};
%  %\node at (t1.north west) {$S$};
%  %\node at (t1) {$D$};
%  %\node[thy,inner sep=.5cm,fill=lightgray] (s2) at (2,0) {};
%  %\node[thy,inner sep=.3cm,fill=gray] (t2) at (2,0) {};
%  %\node at (t2.north west) {$T$};
%  %\node at (t2) {$C$};
%  %\draw[view] (t1) to[out=10,in=170] node[above] {$\sigma$} (t2);
%\end{tikzpicture}
%\end{center}

%\begin{figure}
%\begin{tabular}{c|c}
%\begin{mmtcode}
%Band =
%  include ?SemiGroup
%  axiom_idemp : ⊦ ∀[x] x ∘ x ≐ x
%❚
%
%Regular =
%  include ?Band
%  axiom_regular : ⊦ ∀[x]∀[y]∀[z] 
%    z ∘ x ∘ z ∘ y ∘ z ≐ z ∘ x ∘ y ∘ z
%❚
%
%LeftNormal =
%  include ?Band
%  axiom_leftnormal : ⊦ ∀[x]∀[y]∀[z] 
%    z ∘ x ∘ z ∘ y ≐ z ∘ x ∘ y
%
%theory RightNormal =
%  include ?Band ❙
%  axiom_rightnormal : ⊦ ∀[x]∀[y]∀[z] 
%    y ∘ z ∘ x ∘ z ≐ y ∘ x ∘ z ❙ 
%❚
%
%theory Normal : ?Meta =
%  include ?Band ❙
%  axiom_normal :  ⊦ ∀[x]∀[y]∀[z] 
%    z ∘ x ∘ y ∘ z ≐ z ∘ y ∘ x ∘ z ❙
%❚
%\end{mmtcode} &
%\begin{mmtcode}
%implicit view Reg2LeftNormal : 
%    ?Regular -> ?LeftNormal =
%  include ?Band = ?Band ❙
%  axiom_regular = sketch "trivial" ❙
%❚
%
%implicit view Reg2RightNormal : 
%    ?Regular -> ?RightNormal =
%  include ?Band = ?Band ❙
%  axiom_regular = sketch "trivial" ❙
%❚
%
%implicit view LeftNormal2Normal : 
%    ?LeftNormal -> ?Normal =
%  include ?Band = ?Band ❙
%  axiom_leftnormal = sketch "trivial" ❙
%❚
%
%implicit view RightNormal2Normal : 
%    ?RightNormal -> ?Normal =
%  include ?Band = ?Band ❙
%  axiom_rightnormal = sketch "trivial" ❙
%❚
%\end{mmtcode}
%\end{tabular}
%
%\caption{Exemplary Theories for Some Varieties of Bands}\label{fig:bandsmmt}
%\end{figure}

\subsection{Transparent Refactoring}

A major drawback of using modular theories is that it can preclude transparent refactoring.
For example, consider a theory $t=\{\icl{r},\icl{s}\}$, and assume we want to move a constant declaration $D$ for the name $n$ from $r$ to $s$.
Thus, the change to should be straightforward as it does not change the semantics of $t$.

However, this is not a local change.
It also requires updating every qualified reference from $r?n$ to $s?n$.
Such references can occur anywhere where $t$ is used.
That may include theories that the person who does the refactoring does not know or does not have access to.
Even if the source files always use the unqualified reference $n$ (because the checker is smart enough to dynamically disambiguate them), this still requires a global rebuild to reach a consistent state again.

With implicit morphisms, we can solve this problem by making only the following local changes:
\begin{compactenum}
  \item We rename $s$ to $s'$.
  \item We delete the declaration $n:E$ from $s'$.
  \item We create new theories $r'=\{\icl{r},\,D\}$ and $s=\{\icl{s'},D\}$.
  \item We change $t$ to $t=\{\icl{r'},\icl{s'}\}$.
  \item We add an implicit morphism $s\to t$ that maps $s?n$ to $r'?n$.
\end{compactenum}
Now $t$ has the desired new structure.
But, all old references to $s?n$ stay well-formed so that no global changes are needed.

\ednote{@DM: maybe make a tikz for the before/after situations}

\begin{modexp}
\section{Advanced Module Expressions}\label{sec:complex}
\ednote{this section is outdated}

We now add some complex theories and morphisms to \mmt.
\begin{grammar}
T      & \{\} \alt \bigcup\{\rep{T}\} \alt M(T)   & \text{complex theories}\\
M      & \{\} \alt \bigcup\{\rep{M}\} \alt M^T \alt M^T(M)   & \text{complex morphisms}
\end{grammar}

Their meanings are explained in the sequel.

\paragraph{Empty Theory}
The empty theory $\{\}\in\Thy$ and the empty morphism $\{\}\in\Mo{\{\}}{T}$ for $T\in\Thy$ are always well-formed.
Their semantics is given by $\flt{\{\}}=\es$ (in both cases).
$\{\}$ is an initial object in the category of theories.

The empty morphism is implicit.

\paragraph{Union of Theories}
For any finite set $\Theta\sq\Thy$ of theories, the union theory $\bigcup \Theta\in\Thy$ is always well-formed.
Its semantics is given by $\flt{(\bigcup\Theta)}=\bigcup_{T\in\Theta}\flt{T}$.
As special cases, we can recover $T\equiv\bigcup\{T\}$ and $\{\}\equiv\bigcup\es$.

We write $S\cup T$ for the binary union $\bigcup\{S,T\}$.
Binary union is a semi-lattice (with respect to $\harr$ and $\equiv$) with least element $\{\}$.

Clearly, we have $T\harr \bigcup\Theta$ if $T\in\Theta$.
Moreover, we define the include closure of a set of theories by $\cls{\Theta}=\{C|C\harr T, T\in\Theta\}$ (which is a topological closure operator).
Then we have $\bigcup Z\harr \bigcup \Theta$ iff $\cls{Z}\sq\cls{\Theta}$ and thus also $\bigcup Z\equiv \bigcup \Theta$ iff $\cls{Z}=\cls{\Theta}$.


Now consider morphisms $M_i\in\Mo{S_i}{T_i}$ for $i=1,\ldots,n$.
Let $\mu=\{M_1,\ldots,M_n\}$, $Z=\{S_1,\ldots,S_n\}$ and $\Theta=\{T_1,\ldots,T_n\}$.
The union morphism $\bigcup \mu: \bigcup Z \to \bigcup \Theta$ is well-formed if $M_i|_C\equiv M_j|_C$ whenever $C\harr S_i$ and $C\harr S_j$.
Its semantics is given by $\flt{(\bigcup \mu)}=\bigcup_{M\in\mu} \flt{M}$.

$\bigcup \mu$ is implicit if all $M_i$ are.

\paragraph{Pushout along Inclusion}
For morphisms $M\in\Mo{S}{T}$ and $S\harr X$, the pushout $T\harr M(X)$ with the morphism $M^X\in\Mo{X}{M(X)}$ are well-formed if for all $C\harr X$ and $C\harr T$ also $C\harr S$.

In that case, for $u\in\Mo{T}{U}$ and $x\in\Mo{X}{U}$, the universal morphism $M^u(x)\in\Mo{M(X)}{U}$ of the pushout is well-formed if $x|_S\equiv M;u$.

Their semantics is defined by
\[\flt{M(X)}=\flt{T}\cup \{c[:M(t)][=M(d)]\;|\;c[:t][=d]\in\flt{X}\sm\flt{S}\}\]
\[\flt{(M^X)}=\flt{M}\cup\{c:=c\;|\;c\in\dom{X}\sm\dom{S}\}\]
\[\flt{M^u(x)}=\flt{u}\cup\{c:=x(c)\;|\;c\in\dom{X}\sm\dom{S}\}\]

Note that our definition of $\flt{m(X)}$ crucially exploits the condition that $\dom{X}\sm\dom{S}$ and $\dom{T}$ must be disjoint.
Thus, we obtain $\dom{X}\sm\dom{S}\sq\dom{m(X)}$ and $m^X$ maps these identifiers to themselves.

%(Note that if $M$ is implicit, then necessarily $m|_C\equiv\id{C}$.)

$m^X$ is implicit if $m$ is.\ednote{Can this always be implicit? Even if it can, it is probably not desirable. Compare left-/right-neutral element.}
\ednote{Should there be implicit extensions of morphism? E.g., $m^X$ could be implicit relative to $m$, i.e., we write $m$ and it is inferred as $m^X$. Does that yield canonical structures?}
$m^u(x)$ is implicit if $m$ and $u$ are.


%Pushouts are coherent:
%  $(m;n)(X)\equiv n(m(X))$ if the right hand side is defined.
%  If $X\harr Y$, then $m(X)\harr m(Y)$ if $m(Y)$ is defined.
%  $\id{S}(X)\equiv X$ (always defined).
%  $m(S)\equiv S$ (always defined).
%  
%Pushout over the empty morphism is defined iff $X$ and $T$ are disjoint.
%In that case, it is the union. (But not all unions arise in this way.)
%
%Pushout distributes over union:
%  $m(X\cup Y)\equiv m(X)\cup m(Y)$ (equi-defined)
%  $m^{X\cup Y}\equiv m^X\cup m^Y$ (?)
%  
%  does not make sense:
%  $(\bigcup m_i)(X)\equiv \bigcup \{m_i(X)\}$ 
%  $(\bigcup m_i)^X\equiv \bigcup\{m_i^X\}$
%

% Theorem: For every theory $T$, we have $\dom{T}=\dom{t_1}\cup\ldots\cup\dom{t_n}$ for atomic theories $t_i$.

\end{modexp}

\section{Conclusion and Related Work}\label{sec:conc}
\paragraph{Extending Theories from the Outside}

\paragraph{Transparent Refactoring}
A major drawback of OpenMath/\mmt-style qualified identifiers is that it can preclude transparent refactoring.
For example, consider a theory $t=\{\icl{r},\icl{s}\}$ and assume we want to move a symbol $n$ from $r$ to $s$.
This requires changing any qualified reference to $r?n$ (in any theory including $t$) to $s?n$.
That can be very costly, especially in large libraries where the refactorer might not even know of or have write access to all theories importing $t$.

With implicit morphisms, we rename $s$ to $s'$, move $n$ from $s'$ to $r$, and recover $s$ by $s=\{\icl{s'},n\}$.
Now we can change define $t=\{\icl{r},\icl{s'}\}$ and give an implicit morphism from $s$ to $t$, which maps $s?n$ to $r?n$.
All existing references to $s?n$ stay well-formed (and serve as aliases for $r?n$) so that no further changes in theories importing $t$ are needed.



\bibliographystyle{alpha}
\bibliography{macros/pub_rabe,macros/rabe,macros/systems}

\end{document}