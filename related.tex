\subsection{Realms}

\cite{realms} defines a realm (essentially) as a certain cluster of theories:
\begin{compactitem}
\item a set of isomorphic theories $B_1,\ldots,$
\item for each $B_i$, a set of retractable extensions $E^i_1,\ldots,$
\item a theory $F$ including all of the above.
\end{compactitem}
The intuition is that the $B_i$ are alternative definitions of a theory; the $E^i_j$ add definitions and theorems; and $F$ aggregates everything into the outwardly visible interface (the face) of the realm.

A good example is the realm of topological spaces with the $B_i$ corresponding to the various possible definitions (via neighborhoods, open sets, closure operator, etc.).

Users of the realm include only the face and thus (i) gain access to all derived knowledge in the theory of topological spaces, and (ii) do not have to commit to one particular definition.
Indeed, this corresponds more closely to the way how conventional mathematics sees the theory of topological spaces.

\cite{realms} calls for an implementation of realms as a new primitive concept in addition to theories and morphisms.
However, in the presence of implicit morphisms, we do not need it: All we have to do is to designate the isomorphisms and the retraction is implicit and include any one of the $B_i$ into $F$.


\subsection{Canonical Structures in Coq}

\cite{gonthier_packaging}


\subsection{Implicit Conversions in Scala}

The Scala programming language \cite{scala} has a clean implementation of implicit conversions between any two types.
If we think of functions between classes as theory morphisms, this corresponds to implicit theory morphisms.

Contrary to \mmt, a conversion is only available if explicitly imported into the current namespace.
Moreover, Scala does not chain conversions, i.e, the composition of implicit morphisms is not implicit.
Scala's limitations are intentional: Their goal is to prevent programmers from overusing implicit conversions because that can be very confusing to readers.

These limitations are lifted in \mmt's conversions can only act between theories anyway.
Thus, their potential to confuse is smaller.
Moreover, the use of qualified identifiers in \mmt means that readers can always tell where an identifier is implicitly-included from, and the \mmt IDE can always show the implicit morphism along which it is imported.

\ednote{drawback: global data structure}