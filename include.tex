\subsection{Motivation}

\mmt as introduced in \cite{RK:mmt:10} allowed only two declarations in atomic theories: symbol declarations $c$ and structure declarations, which instantiate and import a theory.
(We omitted the latter in Sect.~\ref{sec:mmt}.)
We will now introduce include declarations to \mmt.

We can see include and structure declarations as duals.
Both import a previously declared theory, and both are transitive in the sense that if $T$ imports $S$, it also imports anything $S$ imports.
They differ in the sharing semantics when the same theory is imported multiple times: Multiple includes of $S$ are redundant and introduce a single copy of $S$; multiple structures of $S$ introduce multiple fresh copies of $S$.

Consequently, include declarations are \emph{unnamed} and the importing theory refers to the included symbols using the identifiers from $\dom{S}$.
Structure declarations on the other hand introduce a name in order to form qualified names for the fresh copy of the imported symbols.

\subsection{Definition}

We extend the \mmt grammar with
\begin{grammar}
  Dec & \icl{S} & \text{include a theory into an atomic theory} \\
  Ass & \icl{M} & \text{include a morphism into an atomic morphism}
\end{grammar}

We extend the definitions of $\flt{\Sigma}$ with the case
  \[\flt{(\Sigma,\;\icl{S})} = \flt{\Sigma}\cup\flt{S}\]
and the definition of $\flt{\sigma}$ with the case
  \[\flt{(\sigma,\;\icl{M})}=\flt{\sigma}\cup\flt{M}\]
%  \[\flt{(\Sigma,\;\strc{s}{S})} = \flt{\Sigma}\cup\{t?s/n[:t'][=d'] | n[:t][=d]\in \flt{S}\}\]
%where $t'$ and $d'$ arise from $t$ and $d$ by renaming all symbol identifiers accordingly.
%  \[\flt{(\sigma,\;\strca{M})}=\flt{\sigma}\cup\{t?s/n:=M(\flt{M}\]
In particular, this yields $S\harr t$ whenever $t$ declares an include of $S$.

Includes between theories are always well-formed (assuming the included theory is).
But includes between morphisms are more difficult -- we have to impose two restrictions.
Firstly, an assignment $\icl{M}$ is well-formed in an atomic morphism $m:S\to T$ if $R\harr S$ and $M\in\Mo{R}{T}$.
Its intended effect is that $m$ maps all $c\in\dom{R}$ to $M(c)$.
That yields the second restriction: For any two morphisms $M_i\in\Mo{R_i}{T}$ included into $m$ and any theory $C$ satisfying $C\harr R_1$ and $C\harr R_2$, we require that $M_1|_C\equiv M_2|_C$.
Thus, any two included morphisms must agree on their common domain $C$; that is necessary so that $\flt{m}$ does not contain multiple assignments for the same identifier.

\subsection{Discussion}

The original structure declarations of \mmt are significantly more expressive than include declarations: They allow fine-tuning the sharing by instantiating some symbols of the imported theory.
In particular, they can emulate include declarations as a special case.

But the inclusion relation $S\harr T$ induced by include declarations has a number of practically valuable properties.
Therefore, the author chose to add include declarations even though they do not adding expressive strength.
Indeed, the practical advantages are huge.
For example, after implementing support for inclusions, the author refactored the LATIN atlas \cite{CHKMR:latinabs:11} -- a large case study on formalizing logics using an \mmt style module system.
At this point it contains approximately $2000$ imports, of which $1600$ were changed to inclusions.
\medskip

There are several reasons for these advantages.
Firstly, inclusion is an order relation between theories: Each theory can be included at most once into another one.
That makes the inclusion relation very easy and efficient to maintain.

Secondly, we can form canonical qualified identifiers $s?n$ for identifiers of $s$ included into a theory $t$.
These are canonical identifiers that uniquely identify the included symbols and that are obvious to users.
Users can easily see where a symbol was imported from, but the import chain along which it was imported remains transparent.
%A qualified identifier $q.c$ where $q$ is provided by the import.
%This quickly leads to long names like $p.q.r.c$. Moreover, because the import path is part of the name, refactoring becomes more costly and multiple qualified identifiers may actually be the same.

Thirdly, $\Exp{s}\sq\Exp{t}$ so that the included declarations do not have to be translated from $s$ to $t$.
Consequently, implementations do not have to duplicate the translations: They can store all local declarations with the declaring theory, and use the relation $s\harr t$ to determine which theory may access which declarations.
Tools that duplicate imported declarations on the other hand quickly suffer from exponential blowups when handling large libraries.

Another advantage is that inclusion allows a form of implicit conversion.
As explained in \cite{RK:mmt:10}, we can consider morphisms $M\in\Mo{T}{F}$ as models of $T$ defined in terms of $F$.
For example, consider $T=\cn{Group}$ and $F=\cn{ZFC}$; then concrete group corresponds to morphisms $G\in\Mo{T}{F}$.
If we further assume $\cn{Monoid}\harr \cn{Group}$, then $G|_{\cn{Monoid}}$ is $G$ seen as a monoid.
Moreover, we can easily make the conversion $G\mapsto G|_{\cn{Monoid}}$ implicit without confusing humans or tools.
\medskip

Of course the above advantages are not a new discovery.
Many languages have long used a combination of unnamed and named imports (e.g., inheritance and aggregation in object-oriented programming correspond to includes and structures).
However, the above appears to be the first time that the semantics of include declarations is defined formally in the context of OpenMath-style qualified identifiers.

%\begin{compactitem}
%\item The original identifier $c$. This quickly leads to name clashes.
%\end{compactitem}
%There is one special case, where choosing identifies is easy: If $S\harr T$, e.g., $S$ is included into $T$.
%Then we can use $S.c$ as a qualified identifier.
%
%The reason why $S.c$ works as a qualified identifier is that $S$ can only be included into $T$ once and in a uniquely determined way.
%Implicit morphisms generalize this property: $S$ can only be implicitly mapped into $T$ once and in a that is uniquely determined by the implicit-diagram.
%Consequently, we can use any declaration $c$ of $S$ in $T$ using the identifier $S.c$.
