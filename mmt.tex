The \mmt language \cite{RK:mmt:10} provides an extremely general setting in which we can define theories.
The precise definitions of the well-formed theories $T$ and theory morphisms $M$ on the one hand and their semantics $\flt{T}$ and $\flt{M}$ on the other hand are \textbf{mutually recursive}.
Therefore, we first give an overview and then recap the definitions below.

\textbf{Flat theories} are lists of symbol declarations $c[:E][=e]$ where $E$ and $e$ are optional expressions.
\textbf{Flat morphisms} from a theory $S$ to a theory $T$ are lists of symbol assignments $c:=e$ where $c$ is declared in $S$ and $e$ is an expression over $T$.
We write $\dom{T}$ for the set of constant identifiers $c$ in $\flt{T}$ and $\Exp{T}$ for the set of closed expressions using only the symbols $c\in\dom{T}$.
Every morphism $M$ induces a \textbf{homomorphic extension} $M(-):\Exp{S}\to\Exp{T}$, which replaces every $c\in\dom{S}$ in an $S$-expressions with the $T$-expressions $o$ such that $(c:=o) \in \flt{M}$.
Both theories and morphisms are subject to typing conditions that are not essential for our purposes here.

Flat theories and morphisms are rarely constructed in practice --- instead, users often construct larger theories from smaller ones.
We distinguish two kinds of constructions:
\begin{compactitem}
 \item \textbf{Structured theories} contain additional declarations such imports and instantiations or high-level declarations.
 \item \textbf{Theory expressions} are operators that take (normal or theory) expressions as input and return theory expressions.
\end{compactitem}
This distinction is orthogonal to the semantic variety, and both kinds are equally expressive, e.g., we can construct the union of two theories using either construction.
For example, we can construct the theory of commutative groups as the structured theory containing the declarations $\icl{Group},\icl{CommutativeMagma}$ or as the theory expression $\cn{Group}\cup\cn{CommutativeMagma}$.
All of the above applies analogously to structured \emph{morphisms} and \emph{morphism} expressions.

Despite the syntactic difference, the semantics of both kinds of constructions can be defined by uniformly by mapping structured theories/theory expressions $T$ to flat theories $\flt{T}$.
We call this mapping \textbf{flattening}.

An MMT \textbf{diagram} consists of a set of structured theory/morphism declarations and induces sets of theory and morphism expressions.
The names of the former are the \textbf{atomic theories/morphisms}, and they provide the base case of the inductive construction of the latter.
Vice versa, the latter may already occur in the bodies of the structured declarations.
Thus, the two concepts must be defined by mutual recursion.

More formally, every diagram induces sets
\begin{compactitem}
 \item $\thy$ of identifiers of atomic theories $t$,
 \item $\Thy\supseteq \thy$ of theory expressions $T$,
 \item $\mo{S}{T}$ of identifiers of atomic morphisms $m$ from $S$ to $T$, and
 \item $\Mo{S}{T}\supseteq\mo{S}{T}$ of morphism expressions $M$ from $S$ to $T$.
\end{compactitem}
Then flattening assigns
\begin{compactitem}
 \item to each $T\in\Thy$ the flat theory $\flt{T}$ (seen as a set of declarations),
 \item to each $M\in\Mo{S}{T}$ the morphism $\flt{M}$ from $\flt{S}$ to $\flt{T}$ (seen as a set of assignments).
\end{compactitem}

Within this framework, we can define new kinds of structuring declarations and new kinds of expression-forming operators along with their flattening.
In both cases, the semantics should be \textbf{compositional} in the following sense:
The flattening of a structured theory arises as the union of the flattenings of each declaration in it.
More formally, for a list $D$ of declarations and a declaration $d$, we have $\flt{D,d}=\flt{D}\cup\fltd{\flt{D}}{d}$ where $\fltd{T}{d}$ defines the semantics of declaration $d$ in the context of theory $T$.
Similarly, the flattening of a theory expression such as $f(S,T)$ depends only on the flattenings of its arguments
More formally, e.g., for theory expressions $S$ and $T$ and a binary theory-forming operator $f$, we have $(\flt{f(S,T)}=\flt{f}(\flt{S},\flt{T})$ where $\flt{f}$ is a function defining the semantics of $f$.

\subsection{Syntax}

We use the non-modular variant of \mmt, which does not use imports and instantiations between theories.
The full version can be found in \cite{RK:mmt:10}.

\begin{definition}[Theory]\label{def:theory}
A \textbf{theory} $T$ is of the form
\begin{grammar}
TDec     & t=\{Dec,\ldots,Dec\}  & \text{atomic theory declaration} \\
Dec      & n\opt{:o}\opt{=o}     & \text{symbol declaration}\\
c        & t?n                   & \text{qualified symbol identifiers} \\
o        & c \alt \ldots         & \text{OpenMath objects built from symbols $c$} \\
T        & t                     & \text{theories} 
\end{grammar}

In an atomic theory, each symbol \textbf{name} $n$ may be declared only once, and its \textbf{type} and \textbf{definiens} (if present) must be closed OpenMath objects using only previously introduced symbols.
We omit the remaining productions for objects here.
\end{definition}

These theories are similar to OpenMath content dictionaries \cite{openmath}.
In particular, \mmt uses OpenMath-style qualified symbol identifiers $t?n$.%
\footnote{Of course, the \mmt IDE and web interface parse and display these identifiers only as $n$ (if that is unambiguous), especially if they are used in formulas.
But the interface always knows the qualified identifier and shows it, e.g., on hover.}
Thus, symbols of the same name declared in different atomic theories are always distinguished.

Symbol declarations subsume virtually all basic declarations common in formal systems such as type/function/predicate symbols, axioms, theorems, inference rules, etc.
In particular, theorems can be represented via the propositions-as-types correspondence as declarations $c:F=p$, which establish theorem $F$ via proof $p$.
Similarly, objects subsume virtually all objects common in formal systems such as terms, types, formulas, proofs.

Individual formal languages arise as fragments of \mmt: They single out the well-formed objects by defining the two \mmt-\textbf{judgments} $\der_T o:o'$ (typing) and $\der_T o\equiv o'$ (equality) for every theory $T$ and $o,o'\in\Exp{T}$.
%This is already sufficient to define logical theorems as those formulas $F$ for which there is an object $p$ (the proof) such that $\der_T p:F$.
The details can be found in \cite{rabe:howto:14}.

\begin{definition}[Morphism]\label{def:morphism}
For two theories $S$ and $T$, a \textbf{morphism} $M$ is of the form
\begin{grammar}
MDec   & m:T\to T=\{Ass,\ldots,Ass\}     & \text{atomic morphism declaration}\\
Ass    & c:=o                            & \text{assignment to symbol}\\
M      & m \alt \id{T} \alt M;M          & \text{morphisms} 
\end{grammar}
such that a morphism declaration contains exactly one assignment $c:=o$ for each $c\in\dom{S}$, all of which satisfying $o\in\Exp{T}$.
\end{definition}

Crucially, morphisms \textbf{preserve judgments}, in particular they map theorems to theorems.
Formally, an atomic morphism $m:S\to T$ is \emph{well-formed} if
\begin{compactitem}
\item whenever $c$ has type $t$ in $\flt{S}$ and $c:= o$ in $\flt{m}$, then $\der_T o:m(t)$,
\item whenever $c$ has definiens $d$ in $\flt{S}$ and $c:= o$ in $\flt{m}$, then $\der_T o\equiv m(d)$.
\end{compactitem}
Then we can show that for every $M\in\Mo{S}{T}$ if $\der_S o:o'$ (or $\der_S o\equiv o'$), then $\der_T M(o):M(o')$ (or $\der_T M(o)\equiv M(o')$).
The details can be found in \cite{rabe:howto:14}.
%In particular, this means that $M$ maps $S$-theorems to $T$-theorems.

\begin{definition}[Diagram]
A \textbf{diagram} is a list of atomic theory and morphism declarations:
\begin{grammar}
Dia    & \rep{(TDec \alt MDec)}          & \text{diagrams}\\
\end{grammar}
A diagram is well-formed if each atomic theory/morphism declaration is well-formed relative to the diagram preceding it.
\end{definition}

\subsection{Semantics}

\paragraph{Induced Declarations and Assignments}
For every atomic theory $t=\{\Sigma\}$, we have $t\in\Thy$.
We define $\flt{t}$ by induction on the declarations in $\Sigma$:
 \[\flt{\cdot}=\es\]
 \[\flt{(\Sigma,\;n[:t][=d])}=\flt{\Sigma}\cup\{t?n[:t][=d]\}\]
where $\cdot$ is the empty sequence.
Correspondingly, for every atomic morphism $m:S\to T=\{\sigma\}$, we have $m\in\Mo{S}{T}$ and $\flt{m}$ is defined by induction on the assignments in $\sigma$:
 \[\flt{\cdot}=\es\]
 \[\flt{(\sigma,\;c:=o)}=\flt{\sigma}\cup\{c:=o\}\]

For the remaining morphisms, the semantics is not as trivial but still straightforward.
For the identity, we have $\id{T}\in\Mo{T}{T}$ if $S\in\Thy$, and $\flt{\id{T}}=\{c:=c|c\in\dom{T}\}$.
For the composition, we have $M;N\in\Mo{R}{T}$ if $M\in\Mo{R}{S}$ and $N\in\Mo{S}{T}$, and $\flt{(M;N)}=\{c:=N(M(c))|c\in\dom{R}\}$.

\paragraph{Equality and Inclusion}
We say that $S$ is \textbf{included} into $T$, written $S\harr T$, if $\flt{S}\sq\flt{T}$.
%In that case, there is an inclusion morphism $S\to T$ (which we also denote $S\harr T$) defined by $\flt{m}=\flt{\id{S}}$.
%Clearly, $m(o)=o$ for all $S$-objects $o$.
Thus, every declaration of $S$ is also available in $T$ and $\Exp{S}\sq\Exp{T}$.
This induces an inclusion morphism, which maps every $c\in\dom{S}$ to itself.
We do not introduce a name for that morphism and simply define $\id{S}\in\Mo{S}{T}$ whenever $S\harr T$ (which includes the special case $S=T$).
Moreover, if $S\harr T$ and $M\in\Mo{T}{U}$, we define the restriction $M|_S\in\Mo{S}{U}$ of $M$ to $S$ by $\id{S};M$.

We say that two theories $S,T\in\Thy$ are \textbf{equal}, written $S\equiv T$, if $\flt{S}=\flt{T}$.
This is equivalent to $S\harr T$ and $T\harr S$.

Correspondingly, we say that two morphisms $M,N\in\Mo{S}{T}$ are \textbf{equal}, written $M\equiv N$, if $\flt{M}=\flt{N}$.
In that case, $\der_T M(o)\equiv N(o)$ for all well-formed $S$-objects $o\in\Exp{S}$.

Clearly, $\equiv$ is an equivalence relation and $\harr$ is an order relation (if anti-symmetry is taken with respect to $\equiv$).
Identity and composition are congruent with respect to $\equiv$.
Moreover, composition is associative and identity neutral with respect to $\equiv$.
Thus, each diagram yields the structure of a category.

\paragraph{Preview}
The above recap of (the non-modular fragment of) \mmt deviates from \cite{RK:mmt:10}, very carefully wording the definitions in a way that makes stating the results of this paper most elegant.
Thus, we intentionally introduced some peculiarities:
\begin{compactitem}
\item We have $\Thy=\thy$ because there are no non-atomic theories.
We will introduce some in Sect.~\ref{sec:complex}.
\item The definitions of $\flt{t}$ and $\flt{m}$ are overly complicated.
That will pay off when we introduce additional declarations in Sect.~\ref{sec:include}: We can define their semantics simply by adding a case to the inductions in $\flt{\Sigma}$ and $\flt{\sigma}$.
\item $s\harr t$ never holds because all $c\in\dom{s}$ are of the form $s?n$ and all $c\in\dom{t}$ are of the form $t?n$.
In fact, $\flt{S}$ and $\flt{T}$ are disjoint whenever $S\neq T$.
That will change when we introduce include declarations in Sect.~\ref{sec:include}.
\end{compactitem}
