\subsection{Overview}

\begin{modexp}
\paragraph{Modularity in MMT}
As much as possible, theories and morphisms are treated uniformly, and we use the word \textbf{module} to refer to them collectively.
\mmt allows constructing large modules from small ones in two ways:
\begin{compactitem}
  \item \textbf{structured modules} contain declarations that import other, previously defined ones,
  \item \textbf{module expressions} are anonymous expressions (akin to formulas, terms, etc.) that denote modules.
\end{compactitem}

These two kinds of constructions have very different advantages but the same expressivity.
For example, we can construct the theory of commutative groups as the structured theory $\cn{CommutativeGroup}=\{\icl{Group},\icl{CommutativeMagma}\}$ or as the theory expression $\cn{Group}\cup\cn{CommutativeMagma}$.
The former is more practical when building named toplevel theories in software systems because each include is a separate declaration that can be parsed, checked, and flattened individually.
While negligible in this example, this becomes relevant quickly in large theories with complex structure.
The latter is more practical when building anonymous theory that are only needed temporarily.
In software systems, they also allow reusing operations like equality and $\lambda$-abstraction that already be implemented at the level of the base logic.
Building large morphism is as important as building large theories: for example, we want to be able to build morphisms out of $\cn{CommutativeGroup}$ by combining compatible morphisms out of $\cn{Group}$ and $\cn{CommutativeMagma}$.
\end{modexp}

\paragraph{Flat Modules}
\textbf{Flat theories} are lists of constant declarations $c:E[=e]$ where $E$ and $e$ are expressions, and the latter is optional.
We write $\dom{T}$ for the set of constant identifiers $c$ in $T$.
Every theory induces the set $\Exp{T}$ for the set of closed \textbf{expressions} using only the symbols $c\in\dom{T}$.
Constant declarations subsume virtually all basic declarations common in formal systems such as type/function/predicate symbols, axioms, theorems, inference rules, etc.
In particular, theorems can be represented via the propositions-as-types correspondence as declarations $c:F=P$, which establish theorem $F$ via proof $P$.
Similarly, \mmt expressions subsume virtually all objects common in formal systems such as terms, types, formulas, proofs.

\textbf{Flat morphisms} from a theory $S$ to a theory $T$ are lists of assignments $c:=e$ where $c\in\dom{S}$ and $e\in\Exp{T}$.
Every morphism $M$ induces a \textbf{homomorphic extension} $M(-):\Exp{S}\to\Exp{T}$, which replaces every $c\in\dom{S}$ in an $S$-expression with the $T$-expression $e$ such that $c:=e$ in $M$.

A \textbf{diagram} consists of a set of theory and morphism declarations.
\begin{nomodexp}
For a given diagram, we write $\Thy$ for the set of declared \textbf{theories}.
For theories $S,T\in\Thy$, we write $\Mo{S}{T}$ for the set of \textbf{morphisms} defined by
\begin{compactitem}
 \item for every declaration $m:S\to T=\{\cdot\}$, we have $m\in\Mo{S}{T}$,
 \item for every $T\in\Thy$, we have $\id{T}\in\Mo{T}{T}$,
 \item for every $M\in\Mo{R}{S}$ and $N\in\Mo{S}{T}$, we have $M;N\in\Mo{R}{T}$ (where $M;N$ corresponds to the usual function composition).
\end{compactitem}
$\Thy$ and $\Mo{S}{T}$ form the category of theories and morphisms.
\end{nomodexp}
\begin{modexp}
Every diagram induces sets of anonymous module expressions.
These are mutually recursive: the names of the former provide the base cases for inductively forming the latter, and the latter may occur in the bodies of the former.
Every diagram induces sets
\begin{compactitem}
 \item $\thy$ of identifiers of theories $t$,
 \item $\Thy\supseteq \thy$ of theory expressions $T$,
 \item $\mo{S}{T}$ of identifiers of atomic morphisms $m$ from $S$ to $T$, and
 \item $\Mo{S}{T}\supseteq\mo{S}{T}$ of morphism expressions $M$ from $S$ to $T$.
\end{compactitem}
\end{modexp}

The theories and morphisms in a diagram may be structured (e.g., by using \emph{include} declarations), and \mmt defines their semantics via \textbf{flattening}, which defines for each $T\in\Thy$ the flat theory $\flt{T}$, and for each $M\in\Mo{S}{T}$ the flat morphism $\flt{M}$ from $\flt{S}$ to $\flt{T}$.

\paragraph{Logics and Well-Formed Expressions}
The logic and the definition of well-formed expressions are not a primary interest of this paper, and we only recap the essential structure needed in the sequel.
We refer to \cite{rabe:howto:14} for details.

\mmt is independent of the base logic and provides a theory and theory morphism layer on top of an arbitrary declarative language.
Individual logics arise as fragments of \mmt: they single out the well-formed expressions by defining the \textbf{judgment} $\der_T e=e':E$ for typing and equality.
(We treat the plain typing judgment $\der_T e:E$ as the special case $\der_T e=e:E$.)
%This is already sufficient to define logical theorems as those formulas $F$ for which there is an object $p$ (the proof) such that $\der_T p:F$.
The declarations in theories and assignments in morphisms are subject to typing conditions, and the main theorem about \mmt is that well-typed morphisms $M:S\to T$ preserve all judgments, i.e., if $\vdash_S e=e':E$, then $\vdash_T M(e)=M(e'):M(E)$.
This includes the preservation of truth via the propositions-as-types principle if $E$ is a proposition and $e$ its proof.

We do not give all rules for these judgments.
The only rule that is relevant to our purposes here is the one for constants:
\[\rul{c:E[=e] \minn \flt{T}}{\vdash_T c[=e]:E}\]
(Here, we merge the two cases where $c$ has a/no definiens $e$ into one rule for convenience.)
Correspondingly, the definition of the homomorphic extension of a morphism with domain $S$ includes the following case for constants:
\[M(c) = \cas{e\mifc (c:E) \in \flt{S}, (c:=e)\in\flt{M} \\ M(e) \mifc (c:E=e)\in \flt{S}}\]
Here if $c$ has a definiens in $S$, we expand it before applying $M$.%
\footnote{The \mmt tool accepts $(c:=e)\in \flt{M}$ even if $(c:E=e')\in\flt{S}$. In that case, \mmt checks $\vdash_T M(e')=e:M(E)$ and puts $M(c)=e$. This is important for efficiency but not essential for our purposes here.}

These base cases introduce a mutual recursion between well-formedness and flattening:
the well-formedness of a declaration in a theory depends on the flattening of all preceding declarations;
correspondingly, the well-formedness of an assignment in a morphism depends on the homomorphic extension of the morphism obtained by flattening of all preceding assignments.
Vice versa, well-formedness is a precondition for defining the flattening --- the definition of flattening may become nonsensical if applied to ill-formed modules.
It might be desirable to define well-formedness independently of flattening.
But our definition captures the typical behavior of practical systems, which first parse, check, and flatten one declaration entirely before moving on to the next one.

Therefore, we will make flattening a partial function, i.e., $\flt{X}$ is undefined if the module $X$ is not well-formed. 

%Consider any declaration $n[:E][=e]$ in the body of a theory $t$.
%It is well-formed if $n$ is a fresh name $E$ and $e$ are well-formed expressions over $t$.
%(Additionally, $E$ and $e$ must satisfy typing conditions that are not essential for our purposes.)
%Importantly, $E$ and $e$ must be from $\Exp{\flt{t}}$ using only constants that were introduced in preceding declarations.
%
%Similarly, consider an assignment $c:=e$ in a morphism $m$ from $S$ to $T$, where $c:E$ is a declaration is $S$.
%It is well-formed if $\vdash_T e:\flt{m}(E)$.

\subsection{Syntax}

We start with the syntax for theories (which arises as a special case of the one given in \cite{RK:mmt:10}):

\begin{definition}[Theory]\label{def:theory}
The grammar for theories and expressions is
\begin{grammar}
TDec     & T=\{Dec,\ldots,Dec\}  & \text{theory declaration} \\
Dec      & n:E\opt{=E}           & \text{constant declaration}\\
c        & T?n                   & \text{qualified constant identifiers} \\
E        & c \alt \ldots         & \text{expressions built from constants} \\
\end{grammar}

In a theory declaration, each symbol \textbf{name} $n$ may be declared only once, and its \textbf{type} and \textbf{definiens} (if present) must be closed expressions over the previously introduced constants only.
We omit the productions and typing rules for the remaining expressions, which include application, binding, variables, literals, etc.
\end{definition}

\begin{example}\label{ex:thgroup}
For the purposes of our running examples, we assume a fixed base logic that provides a simple type system.
\texttt{type} is the universe of types, \lstinline|A|$\rightarrow$\lstinline|B| is the type of functions, and lambda abstraction is written \lstinline|[x:A] t x|.
We also use \mmt notations, which are attached to constant declarations as \lstinline|# <notation>|; these are omitted from the formal grammar above because we only use them in the examples.

Then the (flat) theories $\cn{Group}$ and $\cn{DivGroup}$ from the introduction are:
\begin{mmtcode}
Group =
  U       : type
  op      : U $\to$ U $\to$ U	 # 1 $\circ$ 2 
  unit    : U
  inverse : U $\to$ U   # 1 $^{-1}$ 
  // axioms omitted
\end{mmtcode}
\begin{mmtcode}
DivGroup =
  U   : type
  div : U $\to$ U $\to$ U  # 1 / 2 
  unit: U
  // axioms omitted
\end{mmtcode}
\end{example}

%
%These theories are similar to OpenMath content dictionaries \cite{openmath}.
%In particular, \mmt uses OpenMath-style qualified symbol identifiers $t?n$.%
%\footnote{Of course, the \mmt IDE and web interface parse and display these identifiers only as $n$ (if that is unambiguous), especially if they are used in formulas.
%But the interface always knows the qualified identifier and shows it, e.g., on hover.}
%Thus, symbols of the same name declared in different atomic theories are always distinguished.
%
We proceed accordingly for flat morphisms:

\begin{definition}[Morphism]\label{def:morphism}
The grammar for \textbf{morphisms} is
\begin{grammar}
MDec   & m:T\to T=\{Ass,\ldots,Ass\}     & \text{flat morphism declaration}\\
Ass    & c:=E                            & \text{assignment to symbol}\\
M      & m \alt \id{T} \alt M;M          & \text{morphism expressions}
\end{grammar}
A morphism must contain exactly one assignment $c:=e$ for each $(c:E)\in\flt{S}$, all of which must satisfy $\der_T e: M(E)$.
\end{definition}

\begin{example}[Morphisms]\label{ex:dg2g}
We give the morphism \cn{DG2G} between the theories from Ex.~\ref{syn:incl}:
\begin{mmtcode}
DG2G : DivGroup -> Group =
  U     := U
  div 	:= [a,b] a $\circ$ (b$^{-1}$) 
  unit 	:= unit
  // assignments to axioms omitted
\end{mmtcode}
It maps the universe and unit of a division group to the corresponding notions of a group.
And we have $\cn{DG2G}(a/b)=a\circ b^{-1}$.
Additionally, the morphism maps every axiom of \cn{DivGroup} to a proof in \cn{Group} of the translated statement, but we omit those assignments.
\end{example}

%Crucially, morphisms \textbf{preserve judgments}, in particular they map theorems to theorems.
%Formally, an atomic morphism $m:S\to T$ is \emph{well-formed} if
%\begin{compactitem}
%\item whenever $c$ has type $t$ in $\flt{S}$ and $c:= o$ in $\flt{m}$, then $\der_T o:m(t)$,
%\item whenever $c$ has definiens $d$ in $\flt{S}$ and $c:= o$ in $\flt{m}$, then $\der_T o\equiv m(d)$.
%\end{compactitem}
%Then we can show that for every $M\in\Mo{S}{T}$ if $\der_S o:o'$ (or $\der_S o\equiv o'$), then $\der_T M(o):M(o')$ (or $\der_T M(o)\equiv M(o')$).
%The details can be found in \cite{rabe:howto:14}.
%%In particular, this means that $M$ maps $S$-theorems to $T$-theorems.

Finally, we define diagrams as collections of modules:

\begin{definition}[Diagram]
A \textbf{diagram} is a list of theory and morphism declarations:
\begin{grammar}
Dia    & \rep{(TDec \alt MDec)}          & \text{diagrams}\\
\end{grammar}
Each theory/morphism declaration must have a unique name and be well-formed relative to the diagram preceding it.
\end{definition}

At this point, the grammar only allows forming flat theories and morphisms.
\begin{modexp}
It is intentionally independent of the concrete choice of modularity principles, i.e., the kinds of module-structuring declarations (e.g., includes) and the module-expression-forming operators (e.g., union).
But it already introduces all non-terminals for modularity: new declaration kinds for structured theories arise by adding productions for the non-terminal $Dec$ and new theory expressions are formed by adding productions for the non-terminal $T$.
\end{modexp}
Structuring features can be added to the language incrementally and individually.
\begin{modexp}
We give some examples how to add modularity principles to the grammar:
\end{modexp}

\begin{example}[Includes]\label{syn:incl}
We extend the grammar with
\begin{grammar}
  Dec & \icl{T} & \text{include a theory into a theory} \\
  Ass & \icl{M} & \text{include a morphism into a morphism}
\end{grammar}
This allows writing the theory $\cn{Group}$ from Ex.~\ref{ex:thgroup} by extending a theory of monoids.
Again omitting all axioms, this looks as follows:
\begin{mmtcode}
Monoid =
  U 	: type
  op 	: U $\to$ U $\to$ U	 # 1 $\circ$ 2 
  unit	: U
Group =
  include Monoid
  inverse : U $\to$ U   # 1$^{-1}$ 
\end{mmtcode}
\end{example}

\begin{union}
\begin{example}[Union]\label{syn:union}
We extend the grammar with
\begin{grammar}
  T   & T\cup T & \text{union of theories} \\
  M   & M\cup M & \text{union of morphisms}
\end{grammar}
\end{example}
\end{union}

\begin{modexp}
\begin{example}[Category of Theories]\label{syn:cat}
To obtain the category structure of theories and morphisms, we need identity and composition.
We add them to the grammar as follows:
\begin{grammar}
  M   & \id{T} & \text{identity morphism} \\
  M   & M;M    & \text{morphism composition}
\end{grammar}
\end{example}
\end{modexp}

\subsection{Semantics}

\paragraph{Flattening}
The definition of flattening is \textbf{compositional} in the sense that new modularity principles can be added independently of each other.
\begin{modexp}
In particular, for every module expression $X$, we define its flattening $\flt{X}$ by induction on the syntax.
This follows the principle of compositional semantics, e.g., $\flt{(f(X,Y))}$ depends only on $\flt{X}$ and $\flt{Y}$. 
For theories $T$, at this point, this induction only has the base case of a reference to a declared theory $t$.
Further cases arise when we add structuring declarations.
\end{modexp}

\begin{definition}[Flattening]
For the base case, of a theory $t=\{\Sigma\}$, we define $\flt{t}$ by induction on the declarations in $\Sigma$:
\[\flt{t}=\flt{\Sigma} \tb\mwhere\tb \flt{\cdot}=\es \tb\mand\tb
 \flt{(\Sigma,D)}=\flt{\Sigma}\cup\fltd{\flt{\Sigma}}{D},\]
 where $\cdot$ denotes the empty sequence.
 
Here $\fltd{T}{D}$ is the flattening of declaration $D$ relative to $\flt{\Sigma}$.
At this point, we only have one case for declarations $D$, namely constant declarations.
Their flattening is trivial:
 \[\fltd{\Sigma}{(n:E[=e])}=\{t?n:E[=e]\}\]
where $t$ is the name of the containing theory.

Correspondingly, for a declared morphism $m:S\to T=\{\sigma\}$, $\flt{m}$ is defined by induction on the assignments in $\sigma$:
\[\flt{m}=\flt{\sigma} \tb\mwhere\tb \flt{\cdot}=\es \tb\mand\tb\flt{(\sigma,\;A)}=\flt{\sigma}\cup\fltd{\flt{\sigma}}{A}\]
with the trivial base case
 \[\fltd{\sigma}{(c:=e)}=\{c:=e\}\]
\begin{nomodexp}
Moreover, for $T\in\Thy$ we define $\flt{\id{T}}=\{c:=c \;|\; (c:E)\in\flt{T}\}$
And for $M;N\in\Mo{R}{T}$, we define $\flt{(M;N)}=\{c:=\flt{N}(\flt{M}(c)) \;|\; (c:E)\in\flt{R}\}$.
Note that in both cases, we only have to consider constants without definiens.
\end{nomodexp}
\end{definition}

\begin{example}[Includes (continued from Ex.~\ref{syn:incl})]\label{sem:incl}
The $\keyword{include}$ operator adds new declarations $D$ to a theory, so we have to add cases to the definition of $\fltd{\Sigma}{D}$.
We do this as follows
  \[\fltd{\Sigma}{(\icl{S})} = \flt{S}\]
This has the effect of copying over all declarations from the included into the including theory.

Note that the definition of $\fltd{\Sigma}{(n:E=e)}$ qualifies the constant name $n$ with the theory name $t$ to form the identifier $t?n$ in the flattening.
Therefore, includes can never lead to name clashes.
Also note that, because $\flt{S}$ is a \textit{set} of declarations, the include relation is transitive: if $t$ includes $s$ via two different paths, $\flt{t}$ only contains one copy of the declarations of $\flt{s}$.

The situation is slightly more complicated for morphisms: if a morphism $\sigma$ out of $t$ includes two different morphisms out of $s$, these have to agree.
Therefore, flattening is not always defined:
  \[\fltd{\sigma}{(\icl{M})} = \flt{M}\]
if $\sigma,\icl{M}$ is well-formed and $\flt{M}$ agrees with $\flt{\sigma}$ on any constant that is in the domain of both.
\end{example}

\begin{example}[Continuing Ex.~\ref{syn:incl}]
$\cn{Group}^\flat$ is obtained by copying all included symbols (from $\cn{Monoid}$) over to $\cn{Group}$, resulting in the theory as given in Ex.~\ref{ex:thgroup} except that the identifiers are now, e.g., $\cn{Monoid}?U$ instead of $\cn{Group}?U$.
\end{example}

\begin{union}
\begin{example}[Union (continued from Ex.~\ref{syn:union})]\label{sem:union}
We define the semantics of unions of theories as
\[\flt{(S\cup T)}=\flt{S}\cup\flt{T}\]
The handling of name clashes is the same as for includes:
Because the theories declared in a diagram have unique names and declarations in the flattening are qualified by theory names, union theories cannot have name clashes.
But we have to watch out when taking unions of morphisms:
We define $M_1\cup M_2\in \Mo{S_1\cup S_2}{T}$ if $M_i\in\Mo{S_i}{T}$ and $M_1$ and $M_2$ agree for every constant in $\dom{\flt{S_1}}\cap\dom{\flt{S_2}}$.
In that case, we put
\[\flt{(M\cup N)}=\flt{M}\cup\flt{N}\]
\end{example}
\end{union}

%Includes between theories are always well-formed (assuming the included theory is).
%But includes between morphisms are more difficult -- we have to impose two restrictions.
%Firstly, an assignment $\icl{M}$ is well-formed in an atomic morphism $m:S\to T$ if $R\harr S$ and $M\in\Mo{R}{T}$.
%Its intended effect is that $m$ maps all $c\in\dom{R}$ to $M(c)$.
%That yields the second restriction: For any two morphisms $M_i\in\Mo{R_i}{T}$ included into $m$ and any theory $C$ satisfying $C\harr R_1$ and $C\harr R_2$, we require that $M_1|_C\equiv M_2|_C$.
%Thus, any two included morphisms must agree on their common domain $C$; that is necessary so that $\flt{m}$ does not contain multiple assignments for the same identifier.

\begin{modexp}
\subsection{Semantic Properties}

For two theories $S,T\in\Thy$, we say that $S$ is \textbf{included} into $T$, written $S\harr T$, if $\flt{S}\sq\flt{T}$.
%Clearly, $m(o)=o$ for all $S$-objects $o$.
Thus, every declaration of $S$ is also available in $T$ and $\Exp{S}\sq\Exp{T}$.
This induces an inclusion morphism, which maps every $c\in\dom{S}$ to itself.
We do not introduce a name for that morphism and simply define $\id{S}\in\Mo{S}{T}$ whenever $S\harr T$ (which includes the special case $S=T$).
Moreover, if $S\harr T$ and $M\in\Mo{T}{U}$, we define the restriction $M|_S\in\Mo{S}{U}$ of $M$ to $S$ by $\id{S};M$.

We say that two theories $S,T\in\Thy$ are \textbf{equal}, written $S\equiv T$, if $\flt{S}=\flt{T}$.
This is equivalent to $S\harr T$ and $T\harr S$.

Correspondingly, we say that two morphisms $M,N\in\Mo{S}{T}$ are \textbf{equal}, written $M\equiv N$, if $\flt{M}=\flt{N}$.
In that case, $M(e)=N(e)$ for all expressions $e\in\Exp{S}$.

Clearly, $\equiv$ is an equivalence relation and $\harr$ is an order relation (if anti-symmetry is taken with respect to $\equiv$).

\begin{example}[Includes (continued from Ex.~\ref{sem:incl})]\label{rel:incl}
We have the following soundness theorem for includes: if a theory is declared as $t=\{\Sigma\}$ and $\Sigma$ contains $\icl{S}$, then $S\harr t$.

Note that in the absence of include declarations or unions, $s\harr t$ never holds because all $c\in\dom{s}$ are of the form $s?n$ and all $c\in\dom{t}$ are of the form $t?n$.
In fact, $\flt{s}$ and $\flt{t}$ unless $s=t$.
\end{example}

\begin{union}
\begin{example}[Union (continued from Ex.~\ref{sem:union})]\label{rel:union}
We have the following soundness theorem for unions: $S_i\harr S_1\cup S_2$.
Moreover, $S_1\cup S_2$ is a colimit in the category of theories and morphism, and given $M_1\in\Mo(S_1,T)$ and $M2\in\Mo{S_2,T}$ we have the universal morphism $M_1\cup M_2\in\Mo{S_1\cup S_2}{T}$ out of it.
\end{example}
\end{union}

\begin{example}[Category of Theories (continued from Ex.~\ref{sem:cat})]\label{rel:cat}
Identity and composition are congruent with respect to $\equiv$.
Moreover, composition is associative and identity neutral with respect to $\equiv$.
Thus, each diagram indeed yields the structure of a category with objects $\Thy$ and arrows $\Mo{S}{T}$.
\end{example}
\end{modexp}