\ednote{@Dennis: can you formally work through the example from the intro in here? That would be several example environments for (i) after introducing the syntax of theories, the theory Group and DivGroup (one in detail, one sketched), (ii) after introducing morphisms the two isomorphisms (one in detail, one sketched), (iii) after introducing includes, the refactoring of Group using an include, (iv) examples for how the refactored Group is flattened}

\subsection{Overview}

\paragraph{Modularity in MMT}
As much as possible, theories and morphisms are treated uniformly, and we use the word \textbf{module} to refer to them collectively.
\mmt allows constructing large modules from small ones in two ways:
\begin{compactitem}
  \item \textbf{structured modules} contain declarations that import other, previously defined ones,
  \item \textbf{module expressions} are anonymous expressions (akin to formulas, terms, etc.) that denote modules.
\end{compactitem}

These two kinds of constructions have very different advantages but the same expressivity.
For example, we can construct the theory of commutative groups as the structured theory $\cn{CommutativeGroup}=\{\icl{Group},\icl{CommutativeMagma}\}$ or as the theory expression $\cn{Group}\cup\cn{CommutativeMagma}$.
The former is more practical when building named toplevel theories in software systems because each include is a separate declaration that can be parsed, checked, and flattened individually.
While negligible in this example, this becomes relevant quickly in large theories with complex structure.
The latter is more practical when building anonymous theory that are only needed temporarily.
In software systems, they also allow reusing operations like equality and $\lambda$-abstraction that already be implemented at the level of the base logic.
Building large morphism is as important as building large theories: for example, we want to be able to build morphisms out of $\cn{CommutativeGroup}$ by combining compatible morphisms out of $\cn{Group}$ and $\cn{CommutativeMagma}$.

\paragraph{Flat Modules}
\mmt provides formal syntax for both structured modules and module expressions and defines their semantics via \textbf{flattening}, which defines for every module $M$ a flat module $\flt{M}$.

\textbf{Flat theories} are lists of declarations $c[:E][=e]$ where $E$ and $e$ are optional expressions.
We write $\dom{T}$ for the set of constant identifiers $c$ in $\flt{T}$ and $\Exp{T}$ for the set of closed expressions using only the symbols $c\in\dom{T}$.
Constant declarations subsume virtually all basic declarations common in formal systems such as type/function/predicate symbols, axioms, theorems, inference rules, etc.
In particular, theorems can be represented via the propositions-as-types correspondence as declarations $c:F=P$, which establish theorem $F$ via proof $P$.
Similarly, \mmt expressions subsume virtually all objects common in formal systems such as terms, types, formulas, proofs.

Individual formal languages arise as fragments of \mmt: they single out the well-formed expressions by defining the two \mmt-\textbf{judgments} $\der_T e:e'$ (typing) and $\der_T e\equiv e'$ (equality) for every theory $T$ and $e,e'\in\Exp{T}$.
%This is already sufficient to define logical theorems as those formulas $F$ for which there is an object $p$ (the proof) such that $\der_T p:F$.
The details can be found in \cite{rabe:howto:14}.

\textbf{Flat morphisms} from a theory $S$ to a theory $T$ are lists of assignments $c:=e$ where $c\in\dom{S}$ and $e\in\Exp{T}$.
Every morphism $M$ induces a \textbf{homomorphic extension} $M(-):\Exp{S}\to\Exp{T}$, which replaces every $c\in\dom{S}$ in an $S$-expression with the $T$-expression $e$ such that $c:=e$ in $M$.

The declarations in theories and morphisms are subject to typing conditions that are not essential for our purposes here.
We only mention the main theorem that guarantees that well-typed morphisms preserve judgments, i.e., if $\vdash_S e:E$, then $\vdash_T M(e):M(E)$.
This includes the preservation of truth via the propositions-as-types principle where $E$ is a proposition and $e$ its proof.

An \mmt \textbf{diagram} consists of a set of named structured module declarations and induces sets of anonymous module expressions.
These are mutually recursive: the names of the former provide the base cases for inductively forming the latter, and the latter may occur in the bodies of the former.
Every diagram induces sets
\begin{compactitem}
 \item $\thy$ of identifiers of theories $t$,
 \item $\Thy\supseteq \thy$ of theory expressions $T$,
 \item $\mo{S}{T}$ of identifiers of atomic morphisms $m$ from $S$ to $T$, and
 \item $\Mo{S}{T}\supseteq\mo{S}{T}$ of morphism expressions $M$ from $S$ to $T$.
\end{compactitem}
Then flattening assigns
\begin{compactitem}
 \item to each $T\in\Thy$ the flat theory $\flt{T}$,
 \item to each $M\in\Mo{S}{T}$ the morphism $\flt{M}$ from $\flt{S}$ to $\flt{T}$.
\end{compactitem}

\paragraph{Well-Formedness}
The inference system to define well-formed modules, and expressions is not a primary interest of this paper.
Therefore, we only mention those aspects that are relevant later on.

However, one detail will be critical later on:
the inference rules for the well-formedness of expressions include the following base case for constants:
\[\rul{c:E[=e] \minn \flt{T}}{\vdash_T c:E} \tb \rul{c[:E]=e \minn \flt{T}}{\vdash_T c=e}\]
Correspondingly, the definition of the homomorphic extension of a morphism includes the following base case for constants:
\[M(c) = e \mwhere c:= e\in\flt{M}\]

These base cases introduce a mutual recursion between well-formedness and flattening:
the well-formedness of a declaration in a theory depends on the flattening of all preceding declarations;
and the well-formedness of an assignment in a morphism depends on the homomorphic extension of the morphism obtained by flattening of all preceding assignments.
Vice versa, well-formedness is a precondition for defining the flattening.
It is desirable to define well-formedness independently of flattening.
But the mutual recursion makes sense from an implementation perspective: typical tools for formal systems will first parse, check, and flatten the first declaration, then continue with the next declaration.
In fact, often the checking of well-formedness is as expensive as flattening anyway.

Therefore, we will make flattening a partial function, i.e., $\flt{X}$ is defined iff the module $X$ is well-formed. 

%Consider any declaration $n[:E][=e]$ in the body of a theory $t$.
%It is well-formed if $n$ is a fresh name $E$ and $e$ are well-formed expressions over $t$.
%(Additionally, $E$ and $e$ must satisfy typing conditions that are not essential for our purposes.)
%Importantly, $E$ and $e$ must be from $\Exp{\flt{t}}$ using only constants that were introduced in preceding declarations.
%
%Similarly, consider an assignment $c:=e$ in a morphism $m$ from $S$ to $T$, where $c:E$ is a declaration is $S$.
%It is well-formed if $\vdash_T e:\flt{m}(E)$.

\subsection{Syntax}

We start with the syntax for theories (which arises as a special case of the one given in \cite{RK:mmt:10}):

\begin{definition}[Theory]\label{def:theory}
A \textbf{theory} $T$ is of the form
\begin{grammar}
TDec     & t=\{Dec,\ldots,Dec\}  & \text{theory declaration} \\
Dec      & n\opt{:o}\opt{=o}     & \text{constant declaration}\\
c        & t?n                   & \text{qualified constant identifiers} \\
o        & c \alt \ldots         & \text{expressions built from constants} \\
T        & t                     & \text{theory expressions}
\end{grammar}

In a theory declaration, each symbol \textbf{name} $n$ may be declared only once, and its \textbf{type} and \textbf{definiens} (if present) must be closed expressions only previously introduced constants.
We omit the remaining productions for expressions here, which allow forming complex expressions using application, binding, variables, literals, etc.
\end{definition}

\begin{example}\label{ex:thgroup}
The (flat) theory for $\cn{Group}$ and $\cn{DivGroup}$ as presented in the introduction are:
\begin{mmtcode}
theory Group : ?Meta =
  universe	: type		❘ # U ❙
  op		: U ⟶ U ⟶ U	❘ # 1 ∘ 2 ❙
  unit		: U		❘ # e❙
  inverse : U ⟶ U ❘  # 1 ⁻¹ ❙
  
  axiom_assoc		: ⊦ ∀[x]∀[y]∀[z] x ∘ (y ∘ z) ≐ (x ∘ y) ∘ z ❙
  axiom_right_unit	: ⊦ ∀[x] x ∘ e ≐ x ❙
  axiom_left_unit	: ⊦ ∀[x] e ∘ x ≐ x ❙
  axiom_inverse : ⊦ ∀[x] x ∘ (x⁻¹) ≐ e ❙
❚
\end{mmtcode}
\begin{mmtcode}
theory DivisionGroup : ?Meta =
  universe : type ❘ # U prec 1 ❙
  div : U ⟶ U ⟶ U ❘ # 1 / 2 ❙
  nonEmpty : U ❘ # e prec 1 ❙
  // axioms ... ❙
❚
\end{mmtcode}
where $\GS,\US,\RS$ are delimiters and the (optional) \lstinline|# <not>| is an optional notation. \lstinline|?Meta| is an intended meta-theory containing e.g. the symbols $\forall,\doteq$ etc.
\end{example}

At this point, this grammar only allows forming flat theories, i.e., $\Thy=\thy$.
It is intentionally independent of the concrete choice of modularity principles, i.e., the kinds of module-structuring declarations (e.g., includes) and the module-expression-forming operators (e.g., union).
But it already introduces all non-terminals for modularity: new declaration kinds for structured theories arise by adding production for the non-terminal $Dec$ and new theory expressions are formed by adding productions for the non-terminal $T$.
This makes it easy to incrementally add new declaration kinds and operators to the language.

%These theories are similar to OpenMath content dictionaries \cite{openmath}.
%In particular, \mmt uses OpenMath-style qualified symbol identifiers $t?n$.%
%\footnote{Of course, the \mmt IDE and web interface parse and display these identifiers only as $n$ (if that is unambiguous), especially if they are used in formulas.
%But the interface always knows the qualified identifier and shows it, e.g., on hover.}
%Thus, symbols of the same name declared in different atomic theories are always distinguished.

We proceed accordingly for flat morphisms:

\begin{definition}[Morphism]\label{def:morphism}
A \textbf{morphism} is of the form
\begin{grammar}
MDec   & m:T\to T=\{Ass,\ldots,Ass\}     & \text{flat morphism declaration}\\
Ass    & c:=o                            & \text{assignment to symbol}\\
M      & m                               & \text{morphism expressions}
\end{grammar}
such that a morphism declaration contains exactly one assignment $c:=o$ for each $c\in\dom{S}$, all of which satisfying $o\in\Exp{T}$.
\end{definition}

\begin{example}
	We can now specify the morphisms between our two definitions of groups. The theory of groups as monoids with inverses is given in \autoref{syn:incl}, the theory $\cn{DivGroup}$ and the morphism $\cn{DG2G}$ are as follows:
\begin{mmtcode}
view DG2G : ?DivisionGroup -> ?Group =
  universe = universe ❙
  div = [a,b] a ∘ (b⁻¹) ❙
  nonEmpty = e ❙
  // axioms ...❙
❚
\end{mmtcode}
The morphism $\cn{DG2G}$ hence maps the universe of a division group to the universe of a group, division to the expression $\lambda a,b.\; a\circ b^{-1}$ and the witness $\cn{nonEmpty}$ of a division group to the unit $e$ of a group. Additionally, the view has to map the axioms of a division group to their \emph{proofs} in $\cn{Group}$ of their translated statements.

The reverse morphism $\cn{G2DG}$ is completely analogous, mapping $\cn{op}$, $\cn{unit}$ and $\cn{inverse}$ to their corresponding definitions in terms of the division operation $\cn{div}$.
\end{example}

%Crucially, morphisms \textbf{preserve judgments}, in particular they map theorems to theorems.
%Formally, an atomic morphism $m:S\to T$ is \emph{well-formed} if
%\begin{compactitem}
%\item whenever $c$ has type $t$ in $\flt{S}$ and $c:= o$ in $\flt{m}$, then $\der_T o:m(t)$,
%\item whenever $c$ has definiens $d$ in $\flt{S}$ and $c:= o$ in $\flt{m}$, then $\der_T o\equiv m(d)$.
%\end{compactitem}
%Then we can show that for every $M\in\Mo{S}{T}$ if $\der_S o:o'$ (or $\der_S o\equiv o'$), then $\der_T M(o):M(o')$ (or $\der_T M(o)\equiv M(o')$).
%The details can be found in \cite{rabe:howto:14}.
%%In particular, this means that $M$ maps $S$-theorems to $T$-theorems.

Finally, we define diagrams as collections of modules:

\begin{definition}[Diagram]
A \textbf{diagram} is a list of theory and morphism declarations:
\begin{grammar}
Dia    & \rep{(TDec \alt MDec)}          & \text{diagrams}\\
\end{grammar}
A diagram is well-formed if each atomic theory/morphism declaration has a unique name is well-formed relative to the diagram preceding it.
\end{definition}

We give some examples how to add modularity principles to the grammar:

\begin{example}[Includes]\label{syn:incl}
We extend the grammar with
\begin{grammar}
  Dec & \icl{S} & \text{include a theory into a theory} \\
  Ass & \icl{M} & \text{include a morphism into a morphism}
\end{grammar}
This allows writing the theory $\cn{Group}$ as in \autoref{ex:thgroup} by extending a theory of monoids:
\begin{mmtcode}
theory Monoid : ?Meta =
  universe	: type		❘ # U ❙
  op		: U ⟶ U ⟶ U	❘ # 1 ∘ 2 ❙
  unit		: U		❘ # e❙
  
  axiom_assoc		: ⊦ ∀[x]∀[y]∀[z] x ∘ (y ∘ z) ≐ (x ∘ y) ∘ z ❙
  axiom_right_unit	: ⊦ ∀[x] x ∘ e ≐ x ❙
  axiom_left_unit	: ⊦ ∀[x] e ∘ x ≐ x ❙
❚

theory Group : ?Meta =
  include ?Monoid ❙
	
  inverse : U ⟶ U ❘  # 1 ⁻¹ ❙
  axiom_inverse : ⊦ ∀[x] x ∘ (x⁻¹) ≐ e ❙
❚
\end{mmtcode}
The flat variant $\cn{Group}^\flat$ we get by making all included symbols (from $\cn{Monoid}$) available to $\cn{Group}$, resulting in exactly the theory as presented earlier in \autoref{ex:thgroup}.
\end{example}

\begin{example}[Union]\label{syn:union}
We extend the grammar with
\begin{grammar}
  T   & T\cup T & \text{union of theories} \\
  M   & M\cup M & \text{union of morphisms}
\end{grammar}
\end{example}

\begin{example}[Category of Theories]\label{syn:cat}
To obtain the category structure of theories and morphisms, we need identity and composition.
We add them to the grammar as follows:
\begin{grammar}
  M   & \id{T} & \text{identity morphism} \\
  M   & M;M    & \text{morphism composition}
\end{grammar}
\end{example}

For each example, we will define the semantics in the next section.

\subsection{Semantics}

\paragraph{Flattening}
Just like the syntax, the definition of flattening is \textbf{compositional} in the sense that new modularity principles can be added later independently of each other.
In particular, for every module expression $X$, we define its flattening $\flt{X}$ by induction on the syntax.
This follows the principle of compositional semantics, e.g., $\flt{(f(X,Y))}$ depends only on $\flt{X}$ and $flt{Y}$. 

For theories $T$, at this point, this induction only has the base case of a reference to a declared theory $t$.
Further cases arise when we add structuring declarations.

For the base case, of a theory $t=\{\Sigma\}$, we define $\flt{t}$ by induction on the declarations in $\Sigma$:
\[\flt{t}=\flt{\Sigma}\]
where
 \[\flt{\cdot}=\es\]
 \[\flt{(\Sigma,D)}=\flt{\Sigma}\cup\fltd{\flt{\Sigma}}{D}\]
Here $\fltd{T}{D}$ is the flattening of declaration $d$ in the context of theory $T$.

At this point, we only have one case for declarations $D$, namely constant declarations.
Their flattening is trivial:
 \[\fltd{\Sigma}{n[:E][=e]}=\{t?n[:E][=e]\}\]
where $t$ is the name of the containing theory.

Correspondingly, for a declared morphism $m:S\to T=\{\sigma\}$, we have $m\in\Mo{S}{T}$ and $\flt{m}$ is defined by induction on the assignments in $\sigma$:
 \[\flt{\cdot}=\es\]
 \[\flt{(\sigma,\;A)}=\flt{\sigma}\cup\fltd{\flt{\sigma}}{A}\]
with the trivial base case
 \[\fltd{\sigma}{c:=e}=\{c:=o\}\]

\begin{example}[Includes (continued from Ex.~\ref{syn:incl})]\label{sem:incl}
Includes add new declarations $D$, so we have to add cases to the definition of $\fltd{\Sigma}{D}$.
We do this as follows
  \[\fltd{\Sigma}{\icl{S}} = \flt{S}\]
This has the effect of copying over all declarations from the included into the including theory.

Note that the definition of $\fltd{\Sigma}{n:E=e}$ qualifies the constant name $n$ with the theory name $t$ to form the identifier $t?n$ in the flattening.
Therefore, includes can never lead to name clashes.

Also note, that because $\flt{S}$ is a \textit{set} of declarations, the include relation is transitive: if $t$ includes $s$ via two different paths, $\flt{t}$ only contains one copy of the declarations of $\flt{s}$.
The situation is slightly more complicated for morphisms: if a morphism out of $t$ includes two different morphisms out of $s$, these have to agree.
Therefore, flattening and well-formedness must be distinguished:
  \[\fltd{\sigma}{\icl{M}} = \flt{M}\]
under the condition $\flt{M}$ agrees with $\flt{\sigma}$ on any constant that is in the domain of both.
\end{example}

\begin{example}[Union (continued from Ex.~\ref{syn:union})]\label{sem:union}
We define the semantics of unions of theories as
\[\flt{(S\cup T)}=\flt{S}\cup\flt{T}\]
The handling of name clashes is the same as for includes:
Because the theories declared in a diagram have unique names and declarations in the flattening are qualified by theory names, union theories cannot have name clashes.
But we have to watch out when taking unions of morphisms:
We define $M_1\cup M_2\in \Mo{S_1\cup S_2}{T}$ if $M_i\in\Mo{S_i}{T}$ and $M_1$ and $M_2$ agree for every constant in $\dom{\flt{S_1}}\cap\dom{\flt{S_2}}$.
In that case, we put
\[\flt{(M\cup N)}=\flt{M}\cup\flt{N}\]
\end{example}

Finally, we define the semantics of the category structure:
\begin{example}[Category of Theories (continued from Ex.~\ref{syn:cat})]\label{sem:cat}
For $T\in\Thy$, we put $\id{T}\in\Mo{T}{T}$ and define
\[\flt{\id{T}}=\{c:=c \;|\; c\in\dom{\flt{T}}\}\]

For $R,S,T\in\Thy$ and $M\in\Mo{R}{S}$ and $N\in\Mo{S}{T}$, we put $M;N\in\Mo{R}{T}$ and define
\[\flt{M;N}=\{c:=\flt{N}(\flt{M}(c)) \;|\; c\in\dom{\flt{R}}\}\]
\end{example}

%Includes between theories are always well-formed (assuming the included theory is).
%But includes between morphisms are more difficult -- we have to impose two restrictions.
%Firstly, an assignment $\icl{M}$ is well-formed in an atomic morphism $m:S\to T$ if $R\harr S$ and $M\in\Mo{R}{T}$.
%Its intended effect is that $m$ maps all $c\in\dom{R}$ to $M(c)$.
%That yields the second restriction: For any two morphisms $M_i\in\Mo{R_i}{T}$ included into $m$ and any theory $C$ satisfying $C\harr R_1$ and $C\harr R_2$, we require that $M_1|_C\equiv M_2|_C$.
%Thus, any two included morphisms must agree on their common domain $C$; that is necessary so that $\flt{m}$ does not contain multiple assignments for the same identifier.

\subsection{Semantic Properties}

For two theories $S,T\in\Thy$, we say that $S$ is \textbf{included} into $T$, written $S\harr T$, if $\flt{S}\sq\flt{T}$.
%Clearly, $m(o)=o$ for all $S$-objects $o$.
Thus, every declaration of $S$ is also available in $T$ and $\Exp{S}\sq\Exp{T}$.
This induces an inclusion morphism, which maps every $c\in\dom{S}$ to itself.
We do not introduce a name for that morphism and simply define $\id{S}\in\Mo{S}{T}$ whenever $S\harr T$ (which includes the special case $S=T$).
Moreover, if $S\harr T$ and $M\in\Mo{T}{U}$, we define the restriction $M|_S\in\Mo{S}{U}$ of $M$ to $S$ by $\id{S};M$.

We say that two theories $S,T\in\Thy$ are \textbf{equal}, written $S\equiv T$, if $\flt{S}=\flt{T}$.
This is equivalent to $S\harr T$ and $T\harr S$.

Correspondingly, we say that two morphisms $M,N\in\Mo{S}{T}$ are \textbf{equal}, written $M\equiv N$, if $\flt{M}=\flt{N}$.
In that case, $M(e)=N(e)$ for all expressions $e\in\Exp{S}$.

Clearly, $\equiv$ is an equivalence relation and $\harr$ is an order relation (if anti-symmetry is taken with respect to $\equiv$).

\begin{example}[Includes (continued from Ex.~\ref{sem:incl})]\label{rel:incl}
We have the following soundness theorem for includes: if a theory is declared as $t=\{\Sigma\}$ and $\Sigma$ contains $\icl{S}$, then $S\harr t$.

Note that in the absence of include declarations or unions, $s\harr t$ never holds because all $c\in\dom{s}$ are of the form $s?n$ and all $c\in\dom{t}$ are of the form $t?n$.
In fact, $\flt{s}$ and $\flt{t}$ unless $s=t$.
\end{example}

\begin{example}[Union (continued from Ex.~\ref{sem:union})]\label{rel:union}
We have the following soundness theorem for unions: $S_i\harr S_1\cup S_2$.
Moreover, $S_1\cup S_2$ is a colimit in the category of theories and morphism, and $M_1\cup M_2\in\Mo{S_1\cup S_2}{T}$ is the universal morphism out of it.
\end{example}

\begin{example}[Category of Theories (continued from Ex.~\ref{sem:cat})]\label{rel:cat}
Identity and composition are congruent with respect to $\equiv$.
Moreover, composition is associative and identity neutral with respect to $\equiv$.
Thus, each diagram indeed yields the structure of a category with objects $\Thy$ and arrows $\Mo{S}{T}$.
\end{example}
