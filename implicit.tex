\subsection{Overview}

Our key idea is to use a commutative subdiagram of the \mmt diagram, which we call the \emph{implicit diagram}.
It contains all theories but only some of the morphisms --- the ones designated as \emph{implicit}.
Because the implicit-diagram commutes, there can be at most one implicit morphism from $S$ to $T$ --- if this morphism is $i$, we write $\ipc{i}{S}{T}$.
%Practically, this hasis a multi-graph whose nodes are the theories and whose edges are the atomic morphisms.
%We can extend the correspondence by identifying all paths with the composition of all edges along the path.
%Then concatenation of paths is just composition of morphisms, and the empty path from $T$ to itself is the identity morphism $\id{T}$.
%(This is well-defined because composition is associative and identity is neutral.)
%A diagram is commutative if any two paths between the same nodes are equal (with respect to $\equiv$), i.e., there is at most one morphism between any two nodes.
The implicit-diagram generalizes the inclusion relation:
All identity and inclusion morphisms are implicit, and we recover the inclusion relation $S\harr T$ as the special case $\ipc{\id{S}}{S}{T}$.
And just like inclusion, the relation ``exists $i$ such that $\ipc{i}{S}{T}$'' is a preorder.

Consequently, many of the advantages of inclusions carry over to implicit morphisms:
\begin{compactitem}
\item It is very easy to maintain the implicit-diagram, e.g., as a partial map that assigns to a pair of theories the implicit morphism between them (if any).
\item We can generalize the visibility of identifiers: If $\ipc{i}{S}{T}$, we can use all $S$-identifiers in $T$ as if $S$ were included into $T$.
Any $c\in\dom{S}$ is treated as a valid $T$-identifiers with definiens $M(c)$.
\item We can use canonical identifiers $S?n$ without worrying about ambiguity.
Because there can be at most one implicit morphism $\ipc{i}{S}{T}$, using $S?n$ as an identifier in $T$ is unambiguous.
%Crucially, we can use $S?n$ without bothering what $i$ is.
\end{compactitem}

In practice, the \mmt system searches for an implicit morphism $\ipc{i}{S}{T}$ whenever needed to make an expression well-formed, e.g.,
\begin{compactitem}
 \item An $S$-constant $c$ is used in $T$: treat $c$ as an abbreviation for $i(c)$.
 \item A model $M$ of $T$ is used even though a model of $S$ is expected: reduce $M$ via $i$.
 \item A morphism $M:R\to S$ is composed with a morphism $M':T\to U$: treat $M;M'$ like $M;i;M'$.
\end{compactitem}

%Thus the main question arises what other morphisms we should make implicit.

\subsection{Syntax}

We introduce the family of sets $\IMo{S}{T}$ as a subset of $\Mo{S}{T}$, holding the \textit{implicit} morphisms.
The intuition is that $\Thy$ and $\IMo{S}{T}$ (up to equality of morphisms) form a thin broad subcategory of $\Thy$ and $\Mo{S}{T}$.

It remains to define which morphisms are implicit.
For that purpose, we allow \mmt declarations to carry \textit{attributes}:

\begin{definition}[Attributes for Implicitness]
We add the following productions
\begin{grammar}
MDec   & Att\; MDec  & \text{attributed morphism} \\
Dec    & Att\; Dec   & \text{attributed declaration}\\
Att    & \keyword{implicit} \alt \ldots & \text{attributes}
\end{grammar}
\end{definition}

The set of attributes is itself extensible, and the above grammar only lists the one that we use to get started.
Additional attributes can be added when adding modularity principles.

\begin{example}\label{ex:dg2gimplicit}
We can now change the declaration of the morphism $\cn{DG2G}$ from Ex.~\ref{ex:dg2g} by adding the attribute \keyword{implicit}.
\end{example}

\subsection{Semantics}

We only have to make two minor changes to the semantics to accommodate implicit morphisms.
The first change governs how we obtain implicit morphisms, the second one how we use them.

\begin{definition}[Obtaining Implicit Morphisms]
We define the set $\IMo{S}{T}\sq\Mo{S}{T}$ of \textbf{implicit} morphisms to contain the following elements:
\begin{itemize}
 \item all declared morphisms $m:S\to T=\{\sigma\}$ whose declaration carries the attribute $\keyword{implicit}$,
 \item if $t$ has meta-theory $L$, the identity map as an implicit morphism $L\to t$,
\begin{nomodexp}
 \item all identity morphisms $\id{T}$,
 \item all compositions $M;N$ of implicit morphisms $M$ and $N$,
\end{nomodexp}
 \item all morphisms that additional language features designate as implicit based on the use of additional attributes.
\end{itemize}

Adding an $\keyword{implicit}$ attribute to a declaration is well-formed only if there is (up to equality of morphisms) at most one implicit morphism for any pair of theories.
\end{definition}

\begin{modexp}
Depending on what modularity principles we add, equality of morphisms may or may not be decidable.
Therefore, in the general case, implementations may have to employ sound but incomplete criteria for equality (see Sect.~\ref{sec:commute}).
\end{modexp}

\begin{modexp}
The following example phrases the restriction on implicit morphisms in categorical terms:
\begin{example}[Category of Theories and Implicit Morphisms (continued from Ex.~\ref{sem:cat})]\label{impl:cat}
We define the following morphisms are implicit:
\begin{compactitem}
 \item all identity morphisms $\id{T}$,
 \item all compositions $M;N$ of implicit morphisms $M$ and $N$.
\end{compactitem}
Thus, the implicit morphisms form a thin broad subcategory of theories and morphisms.
In other words, the implicit morphisms form commutative diagram.
\end{example}
\end{modexp}

\begin{example}[Includes (continued from Ex.~\ref{sem:incl})]\label{impl:incl}
For include morphisms, we add the following definition: if a theory $T$ contains the declaration $\icl{S}$, then the induced morphism from $S$ to $T$ is implicit.

Combined with composition morphisms, we see that all transitive includes between theories are implicit.
That corresponds to the intuition that anything that is included is available by its original (i.e. qualified) name.
\end{example}

\begin{example}
	For our morphism $\cn{DG2G}$ from Ex.~\ref{ex:dg2gimplicit}, this means that all symbols declared in the theory $\cn{DivGroup}$ (see Ex.~\ref{ex:thgroup}) are now visible in the theory $\cn{Group}$ with the definitions provided by $\cn{DG2G}$.
	For example, we can write $a/b$ in $\cn{Group}$, where $/$ refers to the identifier $\cn{DivGroup}?\cn{div}$.
	
	(Note that notations are carried over by implicit morphisms as well)
\end{example}

\begin{union}
It is plausible to say that there are implicit morphisms $S_i\harr S_1\cup S_2$ for $i=1,2$ that map all constants to themselves.
However, union expressions do not interact very well with implicit morphisms.
First of all, we have to be careful to maintain the uniqueness condition.
For example, assume there is a implicit morphism $M:S\to T$ that is not simply an inclusion.
Then the union $S\cup T$ cannot be well-formed.
If it were, we would have two different implicit morphisms $S\to S\cup T$, namely $M$ and the inclusion into the union.

Secondly, it becomes difficult to lookup whether an implicit morphism exists between union theories.
We need to consider the equational theory of union (including commutativity, idempotence, and associativity), which is doable for unions but may become undecidable if more module expression formation operators are added.
Already the interaction between union and composition morphisms can be tricky.

Therefore, we use the following definition in our system:

\begin{example}[Union (continued from Ex.~\ref{sem:union})]\label{impl:union}
We assume the definitions from Ex.~\ref{impl:cat} and~\ref{impl:incl} are in place.
We want to add union theories.
First of all, we note that every theory can be normalized into a union of named theories.

Now we make the following restriction on implicit morphisms: we maintain a diagram of implicit morphisms in which all nodes are \emph{named} theories, i.e., we do not allow implicit morphisms between union theories.
Because there are only finitely many named theories, this is a finite data structure.

Consequently, we have to forbid morphism declarations $\keyword{implicit}\,m:S\to T_1\cup T_2$ is a union --- there would be no way to record them in our diagram.
We can allow declarations declarations $\keyword{implicit}\,m:s_1\cup\ldots\cup s_n\to t$ where the domain is a union of named theories --- in that case we simply record $m$ as an implicit morphism from $s_i$ to $t$ for all $i$.
Similarly, any declaration $\icl{s_1\cup\ldots\cup s_n}$ in the body of theory $t$ poses no problems --- we record inclusion morphism $s_1\harr t$.

Now to look up the implicit morphism (if any) from $s_1\cup\ldots\cup s_k\to t_1\cup\ldots\cup t_l$, we look up for every $i$ an implicit morphism $m_i:s_i\to t_j$ for some $j$.
If all these $m_i$ exist and $m=m_1\cup \ldots\cup m_k$ is well-formed, then $m$ is the needed implicit morphism.
\end{example}
\end{union}

The intuition behind implicit morphisms is that all $S$-constants $c:E$ that can be mapped into the current theory via an implicit morphism $M:S\to T$ are directly available in $T$.
We can practically realize this by adding new defined constants $c:M(E)=M(c)$ to $T$.
However, physically adding definitions can be inefficient.
It is more elegant to modify the typing rules such that $\vdash_T c:M(c)=M(E)$ holds without any changes to $T$.

To do that, we only have to make a small modification to the original rules of \mmt as presented in Sect.~\ref{sec:mmt}.
To illustrate how simple the modification is, the following definition repeat the original rule first for comparison:

\begin{definition}[Using Implicit Morphisms]\label{def:useimpl}
We replace the rule
\[\rul{c:E[=e] \minn \flt{T}}{\vdash_T c[=e]:E} \tb \mwith\tb \rul{c:E[=e] \minn \flt{S}\tb \ipc{M}{S}{T}}{\vdash_T c=M(c):M(E)}\]
\end{definition}

Note that the modified rule gives every constant a definiens.
This is a technical trick to subsume the original rule: if $c$ is already declared in $T$, we use $M=\id{T}$ and obtain $\vdash_T c=c:E$.

The following theorem is our central theoretical result.
It shows that the modification made in Def.~\ref{def:useimpl} has the intended properties:
\begin{theorem}[Conservativity of Implicit Morphisms]
For well-formed diagrams and $S,T\in\Thy$ and $M\in\IMo{S}{T}$:
\begin{compactenum}
 \item Whenever the original system proves $\vdash_T e=e':E$, so does the modified one.
 \item Whenever the modified system proves $\vdash_T e=e':E$, then the original system proves $\vdash_T \ov{e}=\ov{e'}:\ov{E}$.
\end{compactenum}
Here $\ov{E}$ is the expression that arises from $E$ by recursively replacing every constant with its definiens.
\end{theorem}

\begin{proof}
For the first claim, we proceed by induction on derivations.
We only need to consider the case where the original rule was applied.
So assume it yields $\vdash_T c[=e]:E$, i.e., $(c:E[=e]) \minn \flt{T}$.
We apply the modified rule for the special case $\ipc{\id{T}}{T}{T}$.
The conclusion reduces to
\begin{compactitem}
 \item if $e$ is absent: $\vdash_T c=c:E$, which is equivalent to $\vdash_T c:E$,
 \item if $e$ is present: $\vdash_T c=e:E$ because $M(c)=M(e)$ according to the definition of $M(-)$.
\end{compactitem}

For the second claim, we proceed by induction on derivations.
We only need to consider the case where the modified rule was applied.
So assume it yields $\vdash_T c=M(c):M(E)$ for $(c:E[=e]) \minn \flt{S}$ and $\ipc{M}{S}{T}$.
We distinguish two cases:
\begin{compactitem}
 \item $M(c)=c$, i.e., $e$ is absent, and $\ov{c}=c$.
   According to the definition of $\flt{S}$, this is only possible if $S\harr T$ (including the special case $S=T$).
   In that case, $\flt{S}\sq\flt{T}$ and $M$ is the include/identity morphism that maps all $S$-constants to themselves.
   Now applying the induction hypothesis to the well-formedness derivation of $E$ yields $\vdash_T c:\ov{E}$ as needed.
 \item $M(c)\neq c$, i.e., $e$ is present or $M$ assigns a non-trivial value to $c$.
  Definition expansion eliminates $c$ in favor of $M(c)$, and thus $\ov{c}=\ov{M(c)}$.
  We only have to show that $\vdash_T \ov{M(c)}:\ov{M(E)}$. That follows from the judgment preservation of morphisms.
\end{compactitem}
\end{proof}


With implicit morphisms, we can also relax the semantics of many structuring features:

\begin{example}[Includes and Meta-Theories (continued from Ex.~\ref{sem:incl})]\label{impl:incl2}
Consider $t:L=\{\ldots,\icl{S},\ldots\}$ where $S$ has meta-theory $L'$.
Instead of requiring $L'\harr L$ as in Ex.~\ref{sem:incl}, we require only $\ipc{i}{L'}{L}$.

In that case, we treat $\icl{S}$ as an abbreviation for $\icl{i(S)}$, where $i(S)$ is the pushout of $S$ along $i$ (see \cite{rabe:howto:14} for details on \mmt pushouts).
If $L'\harr L$, this reduces to the original semantics.
\end{example}

%\subsection{Commutativity}\label{sec:commute}
%
%To prove commutativity, we have to find all paths in the implicit-diagram and check $M\equiv M'$ for all pairs of morphisms between the same nodes.
%
%\mmt reduces $M\equiv M'$ for $M,M'\in\Mo{S}{T}$ to a first-order problem in the theory of categories, which uses the following axioms:
%\begin{compactitem}
%\item associativity of composition,
%\item neutrality of identity,
%\item $M;M'\equiv \id{S}$ as well as $M';M\equiv\id{T}$ if $M\in\Mo{S}{T}$ and $M'$ is the known *inverse of $M$,
%\item if an atomic morphism $m$ includes $M\in\Mo{R}{T}$, then $M|_R\equiv M$.
%\end{compactitem}
%
%This is a sound but not complete axiomatization of $M\equiv M'$.
%We can obtain a complete axiomatization in a totally different way: Check $\der_T M(c)\equiv M'(c)$ for all $c\in\dom{S}$.
%However, this can be extremely expensive: It requires computing $\dom{S}$ and proving an object level equality for each identifier.
%Even if the equality of objects is decidable, this is often too expensive.
%
%In fact, because implicit morphisms are meant to be used in very particular cases, we expect the above incomplete calculus to perform reasonably well in practice.
%If showing the equality of certain morphisms is hard, one should probably not make them implicit in the first place.

