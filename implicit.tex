\subsection{Overview}

The motivation behind implicit morphisms is to generalize the benefits of include declarations even further.
Sometimes an atomic morphism is conceptually similar to an inclusion: For example, we might want to declare $m:\cn{Division}\to\cn{Group}=\{\cn{divide} := \lambda x,y.x\circ y^{-1}, \;\ldots\}$ to show how division is ``included'' into the theory of groups.
Other times we have to use structures because we need to duplicate declarations but we would still like to use them like includes.
For example, we have to declare the theory of rings using two structures (for the commutative group and the monoid).
But we still want a ring to be implicitly converted into, e.g., a commutative group.

Our key idea is to use a commutative{\footnotemark} subdiagram of the \mmt diagram, which we call the \emph{implicit-diagram}.
It contains all theories but only some of the morphisms -- the ones designated as \emph{implicit}.
Because the implicit-diagram commutes, there can be at most one implicit morphism from $S$ to $T$ -- if this morphism is $M$, we write $\ipc{M}{S}{T}$.
\footnotetext{A diagram commutes if for any two morphisms $M,M'\in\Mo{S}{T}$, we have $M\equiv M'$.}
%Practically, this hasis a multi-graph whose nodes are the theories and whose edges are the atomic morphisms.
%We can extend the correspondence by identifying all paths with the composition of all edges along the path.
%Then concatenation of paths is just composition of morphisms, and the empty path from $T$ to itself is the identity morphism $\id{T}$.
%(This is well-defined because composition is associative and identity is neutral.)
%A diagram is commutative if any two paths between the same nodes are equal (with respect to $\equiv$), i.e., there is at most one morphism between any two nodes.

The implicit-diagram generalizes the inclusion relation $\harr$.
In particular, all identity and inclusion morphisms are implicit, and we recover $S\harr T$ as the special case $\ipc{\id{S}}{S}{T}$.
Moreover, the relation ``exists $M$ such that $\ipc{M}{S}{T}$'' is also an order.

Consequently, many of the advantages of inclusions carry over to implicit morphisms.
\begin{compactitem}
\item It is very easy to maintain the implicit-diagram, e.g., as a partial map that assigns to a pair of theories the implicit morphism between them (if any).
\item We can generalize the visibility of identifiers: If $\ipc{M}{S}{T}$, we can use all $S$-identifiers in $T$ as if $S$ were included into $T$.
Any $c\in\dom{S}$ is treated as a valid $T$-identifiers with definiens $M(c)$.
\item We can still use canonical identifiers $s?n$. Because there can be at most one implicit morphism $\ipc{M}{s}{T}$, using $s?n$ as an identifier in $T$ is unambiguous.
Crucially, we can use $s?n$ without bothering what $M$ is.
\end{compactitem}

%Thus the main question arises what other morphisms we should make implicit.

\subsection{Syntax}

We introduce the family of sets $\IMo{S}{T}$ as a subset of $\Mo{S}{T}$, holding the \textit{implicit} morphisms.
The intuition is that $\Thy$ and $\IMo{S}{T}$ (up to equality of morphisms) form a thin broad subcategory of $\Thy$ and $\Mo{S}{T}$.

It remains to define which morphisms are implicit.
For that purpose, we allow \mmt declarations to carry \textit{attributes}.
Precisely, we add the following productions
\begin{grammar}
MDec   & Att\; MDec  & \text{attributed morphism} \\
Dec    & Att\; Dec   & \text{attributed declaration}\\
Att    & \keyword{implicit} \alt \ldots & \text{attributes}
\end{grammar}

The set of attributes is itself extensible, and the above grammar only list the one that we use to get started.
Additional attributes can be added when adding modularity principles.

\begin{example}\label{ex:dg2gimplicit}
We can now change the declaration of the morphism $\cn{DG2G}$ from Ex.~\ref{ex:dg2g} by adding the attribute \keyword{implicit}.
\end{example}

\subsection{Semantics}

We only have to make two minor changes to the semantics to accommodate implicit morphisms.
The first change govern how we obtain implicit morphisms, the second one how we use them.

\paragraph{Obtaining Implicit Morphisms}
Based on the above attributes, we define the set $\IMo{S}{T}\sq\Mo{S}{T}$ of \textbf{implicit} morphisms to contain the following elements:
\begin{itemize}
 \item all declared morphisms $m:S\to T=\{\sigma\}$ whose declaration carries the attribute $\keyword{implicit}$,
\begin{nomodexp}
 \item all identity morphisms $\id{T}$,
 \item all compositions $M;N$ of implicit morphisms $M$ and $N$,
\end{nomodexp}
 \item all morphisms that additional language features designate as implicit based on the use of additional attributes.
\end{itemize}

To define well-formedness for our extended syntax, we say that adding an attribute to a declaration is only well-formed if there is (up to equality of morphisms) at most one implicit morphism for any pair of theories, i.e., $\IMo{S}{T}$ has cardinality $0$ or $1$.
Thus, the implicit morphisms form a thin broad subcategory of theories and morphisms.
In other words, the implicit morphisms form commutative diagram.

\begin{modexp}
Depending on what modularity principles we add, equality of morphisms may or may not be decidable.
Therefore, in the general case, implementations may have to employ sound but incomplete criteria for equality (see Sect.~\ref{sec:commute}).
\end{modexp}

\begin{modexp}
The following example phrases the restriction on implicit morphisms in categorical terms:
\begin{example}[Category of Theories and Implicit Morphisms (continued from Ex.~\ref{sem:cat})]\label{impl:cat}
We define the following morphisms are implicit:
\begin{compactitem}
 \item all identity morphisms $\id{T}$,
 \item all compositions $M;N$ of implicit morphisms $M$ and $N$.
\end{compactitem}
Thus, the implicit morphisms form a thin broad subcategory of theories and morphisms.
In other words, the implicit morphisms form commutative diagram.
\end{example}
\end{modexp}

\begin{example}[Includes (continued from Ex.~\ref{sem:incl})]\label{impl:incl}
For include morphism, we add the following definition: if theory $t$ contains the declaration $\icl{S}$, then the induced morphism from $S$ to $t$ is implicit.

Combined with composition morphisms, we see that all transitive includes between theories are implicit.
That corresponds to the intuition that anything that is include should be available directly.
\end{example}

\begin{example}
	For our morphism $\cn{DG2G}$ from Ex.~\ref{ex:dg2gimplicit}, this means that all symbols declared in the theory $\cn{DivisionGroup}$ (see Ex.~\ref{ex:thgroup}) are now visible in the theory $\cn{Group}$ with the definitions provided by $\cn{DG2G}$.
\end{example}

\begin{union}
It is plausible to say that there are implicit morphisms $S_i\harr S_1\cup S_2$ for $i=1,2$ that map all constants to themselves.
However, union expressions do not interact very well with implicit morphisms.
First of all, we have to be careful to maintain the uniqueness condition.
For example, assume there is a implicit morphism $M:S\to T$ that is not simply an inclusion.
Then the union $S\cup T$ cannot be well-formed.
If it were, we would have two different implicit morphisms $S\to S\cup T$, namely $M$ and the inclusion into the union.

Secondly, it becomes difficult to lookup whether an implicit morphism exists between union theories.
We need to consider the equational theory of union (including commutativity, idempotence, and associativity), which is doable for unions but may become undecidable if more module expression formation operators are added.
Already the interaction between union and composition morphisms can be tricky.

Therefore, we use the following definition in our system:

\begin{example}[Union (continued from Ex.~\ref{sem:union})]\label{impl:union}
We assume the definitions from Ex.~\ref{impl:cat} and~\ref{impl:incl} are in place.
We want to add union theories.
First of all, we note that every theory can be normalized into a union of named theories.

Now we make the following restriction on implicit morphisms: we maintain a diagram of implicit morphisms in which all nodes are \emph{named} theories, i.e., we do not allow implicit morphisms between union theories.
Because there are only finitely many named theories, this is a finite data structure.

Consequently, we have to forbid morphism declarations $\keyword{implicit}\,m:S\to T_1\cup T_2$ is a union --- there would be no way to record them in our diagram.
We can allow declarations declarations $\keyword{implicit}\,m:s_1\cup\ldots\cup s_n\to t$ where the domain is a union of named theories --- in that case we simply record $m$ as an implicit morphism from $s_i$ to $t$ for all $i$.
Similarly, any declaration $\icl{s_1\cup\ldots\cup s_n}$ in the body of theory $t$ poses no problems --- we record inclusion morphism $s_1\harr t$.

Now to look up the implicit morphism (if any) from $s_1\cup\ldots\cup s_k\to t_1\cup\ldots\cup t_l$, we look up for every $i$ an implicit morphism $m_i:s_i\to t_j$ for some $j$.
If all these $m_i$ exist and $m=m_1\cup \ldots\cup m_k$ is well-formed, then $m$ is the needed implicit morphism.
\end{example}
\end{union}

\paragraph{Using Implicit Morphisms}
The intuition behind implicit morphisms is that all $S$-constants $c:E$ that can be mapped into the current theory via an implicit morphism $M:S\to T$ are directly available in $T$.
We can practically realize this by adding new defined constants $c:M(E)=M(c)$ to $T$.
However, physically adding definitions can be inefficient.
It is more elegant to modify the typing rules such that $\vdash_T c:M(c)=M(E)$.

To do that, we only have to make a small modification to the original rules of \mmt as presented in Sect.~\ref{sec:mmt}.
To illustrate how simple the modification is, we repeat the original rule first for comparison:
\[\rul{c:E[=e] \minn \flt{T}}{\vdash_T c[=e]:E}\]
Now our modified rule is
\[\rul{c:E[=e] \minn \flt{S}\tb \ipc{M}{S}{T}}{\vdash_T c=M(c):M(E)}\]

Note that the modified rule gives every constant a definiens.
This is a technical trick to subsume the original rule: if $c$ is already declared in $T$, we use $M=\id{T}$ and obtain $\vdash_T c=c:E$.
We write $\ov{E}$ for the expression that arises from $E$ by recursively replacing every $c$ with its definiens.

The following theorem is our central theoretical result.
It shows that the above modification has the intended properties:
\begin{theorem}[Conservativity of Implicit Morphisms]
Consider any combination of modular features from the examples above that includes at least implicit identity morphisms.

Then for well-formed diagrams and $S,T\in\Thy$ and $M\in\Mo{S}{T}$
\begin{compactenum}
 \item Whenever the original system proves $\vdash_T e=e':E$, so does the modified one.
 \item Whenever the modified system proves $\vdash_T e=e':E$, then the original inference system proves $\vdash_T \ov{e}=\ov{e'}:\ov{E}$.
\end{compactenum}
\end{theorem}

\begin{proof}
For the first claim, we proceed by induction on derivations.
We only need to consider case where the original rule was applied.
So assume it yields $\vdash_T c[=e]:E$, i.e., $(c:E[=e]) \minn \flt{T}$.
We apply the modified rule for the special case $\ipc{\id{T}}{T}{T}$.
The conclusion reduces to
\begin{compactitem}
 \item if $e$ is absent: $\vdash_T c=c:E$, which is equivalent to $\vdash_T c:E$,
 \item if $e$ is present: $\vdash_T c=e:E$ because $M(c)=M(e)$ according to the definition of $M(-)$.
\end{compactitem}

For the second claim, we proceed by induction on derivations.
We only need to consider the cases where the modified rule was applied.
So assume it yields $\vdash_T c=M(c):M(E)$ for $(c:E[=e]) \minn \flt{S}$ and $\ipc{M}{S}{T}$.
We distinguish two cases:
\begin{compactitem}
 \item $M(c)=c$, i.e., $c$ does not have a definiens:
   According to the definition of $M(-)$ this is only possible if $e$ is absent.
   According to the definition of $\flt{T}$, this is only possible if $c=S?n$ for $S\harr T$ (including the special case $S=T$).
   In that case, $\flt{S}\sq\flt{T}$ and $M$ is the include/identity morphism that maps all $S$-constants to themselves.
   Now applying the induction hypothesis to the well-formedness derivation of $E$ yields $\vdash_T c:\ov{E}$ as needed.
 \item $M(c)\neq c$, i.e., $c$ has a non-trivial definiens:
  Definition expansion eliminates $c$ in favor of $e$.
  Clearly $\ov{c}=\ov{M(c)}$, and we only have to show that $\vdash_T \ov{M(c)}:\ov{M(E)}$ That follows from the judgment preservation of morphisms.
\end{compactitem}
\end{proof}

\section{Identifying Theories via Implicit Isomorphisms}\label{sec:inverse}

In this section, we introduce several language extensions that introduce implicit isomorphisms.

Note that because identity morphisms are implicit, our uniqueness requirement for implicit morphisms implies that two theories $S$ and $T$ must be isomorphic if there are implicit morphisms in both directions.
Moreover, making a pair of isomorphisms implicit is well-formed if there are no other implicit morphisms between $S$ and $T$ yet.

\paragraph{Renamings}
We say that a named morphism $r:S\to T=\{\ldots\}$ is a \textbf{renaming} if
\begin{compactitem}
 \item all assignments in its body are of the form $c:=c'$ for $T$-constants $c'$ without definiens
 \item every such $T$-constants $c'$ occurs in exactly one assignment.
\end{compactitem}
Clearly, every renaming is an isomorphism.
The inverse morphisms contains the flipped assignments $c':=c$.

We make the following extension to syntax and semantics:
\begin{compactitem}
  \item A morphism declaration $r:s\to t=\{\ldots\}$ may carry the attribute \keyword{renaming}.
  \item This is well-formed if there are no implicit morphism between $s$ and $t$ yet.
  \item In that case, we define $r\in\IMo{s}{t}$ and $r^{-1}\in\Mo{t}{s}$.
\end{compactitem}

\ednote{@DM: add example, e.g., a definition of Monoid that uses different names}

\paragraph{Definitional Extensions}
We say that the named theory $t$ is a \textbf{definitional extension} of $S$ if $t=S$ or the body of $t$ contains
\begin{compactitem}
 \item only constant declarations with definiens,
 \item only include declarations of theories that are definitional extensions of $S$.
\end{compactitem}
\ednote{@DM: give example that extends Group with the theorem $(x^{-1})^{-1}\doteq x$ (with omitted proof)}

If $t$ is a definitional extension of $S$, it is easy to prove that $t$ and $S$ are isomorphic: both isomorphisms map all constants without definiens to themselves. In particular, the isomorphism $S\to t$ maps $S$-constants to themselves and expands the definiens of all other constants.

We make the following extension to syntax and semantics:
\begin{compactitem}
  \item An include declaration $\icl{s}$ of a named theory $s$ inside theory $t$ may carry the attribute \keyword{definitional}.
  \item In that case, we define $\id{s}\in\IMo{t}{s}$ (in addition to the implicit morphism $\id{t}\in\IMo{s}{t}$ which is induced by the inclusion).
\end{compactitem}

\ednote{@DM: repeat previous example now with \keyword{definitional} and give another definitional extensions of Group; this new extension may already use the implicitly available definition of the first extension}

\begin{remark}[Conservative Extensions]
A definitional extension is a special case of a conservative extension.
More generally, all retractable extensions are conservative, i.e., all extensions $S\harr T$ such that there is a morphism $r:T\to S$ such that $r$ is the identity on $S$.

But we cannot make the retractions implicit morphisms in general because they are not necessarily isomorphisms.
\end{remark}

\paragraph{Canonical Isomorphisms}
If we have isomorphisms $m:s\to t$ and $n:t\to s$, we simply spell them out in morphism declarations and add the keyword \keyword{implicit} to both.
This requires no language extensions.

\ednote{@DM: add the isomorphism of DG2G}
\begin{example}\label{group:iso}
We give the morphism $\cn{G2DG}$.

\end{example}

While the first one of these declarations is straightforward, the second one requires checking that $m$ and $n$ are actually isomorphism.
Otherwise, the uniqueness condition would be violated.
Thus, we have to check $m;n=\id{s}$ and $n;m=\id{t}$.
In general, the equality of two morphisms $f,g:A\to B$ is equivalent to $\vdash_B f(c)=g(c)$ for all $c:E\in\flt{A}$.
Thus, if equality of expressions is decidable in the logic that \mmt is instantiated with, then \mmt can check this directly.

However, this does not work in practice.
Already elementary examples require stronger, undecidable equality relations are used:

\begin{example}
Consider the isomorphism from Ex.~\ref{group:iso}.
The result of mapping $x\circ y$ from $\cn{Group}$ to $\cn{DivGroup}$ and back is $x\circ(unit\circ y^{-1})^{-1}$.
Clearly, the group axioms imply that this is equal to $x\circ y$.
But formally that requires working with the undecidable equality of first-order logic.
\end{example}

Therefore, in our running example, we only make one of the two isomorphisms implicit.

In the sequel, we design a general solution to this problem.
It allows systematically proving the equality of two morphisms and using that to make both isomorphisms implicit.
This is novel work, but it requires significant prerequisites and is only peripherally related to implicit morphisms.
Therefore, we only sketch the idea and leave the details to future work.

We add a language feature to \mmt to prove equalities between morphisms:
We add the productions
\begin{grammar}
Dia   & \rep{(TDec \alt MDec\alt MEq)}  &\\
MEq   & \keyword{equal} M=M:T\to T by \{\rep{Ass}\} &\\
\end{grammar}

%Firstly, the intuition of $\keyword{equality} T = \{c:=E,\ldots\}$ is that it provides to every base type $c:\type$ of $T$ a judgment $E:c\to c\to \type$ that defines the $T$-specific equality relation for objects of type $c$.
%Technically, this must be a binary logical relation in the sense of \cite{RS:logrels:12} on $\id{T}$.
%As described in \cite{RS:logrels:12}, this induces an equality predicate $\cn{Equal}_E:E\to E\to \type$ on every type $E$.

We define the declaration $M=N:S\to T by \{\sigma\}$ to be well-formed iff
\begin{compactitem} 
  \item $M:S\to T$ and $N:S\to T$ are well-formed morphisms
  \item $\sigma$ contains exactly one assignment $c:=p$ for every $(c:E)\in\flt{S}$
  \item for each of these assignments $c:=p$, the term $p$ is a proof of $\vdash_T M(c)=N(c)$.
\end{compactitem}

To make $m$ and $n$ from above implicit isomorphisms, we have to do three things: define $m$ and $n$, prove the equalities of $m;n=\id{s}$ and $n;m=\id{t}$, and make $m$ and $n$ implicit.
Note that we cannot make both $m$ and $n$ implicit right away because that is only well-formed after proving the equalities.)
Thus, we define a new attribute \keyword{implicit-later}, which states that a morphism should be considered implicit as soon as subsequent equality proves make it well-formed to do so.

\begin{example}[Isomorphisms]
We can now add declarations
 \[\keyword{implicit}:\cn{DG2G}:\cn{DivGroup}\to\cn{Group}=\text{(as above)}\]
 \[\keyword{implicit-later}:\cn{G2DG}:\cn{Group}\to\cn{DivGroup}=\text{(as above)}\]
 \[\keyword{equal}\cn{G2DG};\cn{DG2G}=\id{\cn{Group}}:\cn{Group}\to\cn{Group}=\text{(omitted)}\]
 \[\keyword{equal}\cn{DG2D};\cn{G2DG}=\id{\cn{DivGroup}}:\cn{DivGroup}\to\cn{DivGroup}=\text{(omitted)}\]
where the isomorphisms are as above and we omit all the equality proofs.
\end{example}


%\subsection{Commutativity}\label{sec:commute}
%
%To prove commutativity, we have to find all paths in the implicit-diagram and check $M\equiv M'$ for all pairs of morphisms between the same nodes.
%
%\mmt reduces $M\equiv M'$ for $M,M'\in\Mo{S}{T}$ to a first-order problem in the theory of categories, which uses the following axioms:
%\begin{compactitem}
%\item associativity of composition,
%\item neutrality of identity,
%\item $M;M'\equiv \id{S}$ as well as $M';M\equiv\id{T}$ if $M\in\Mo{S}{T}$ and $M'$ is the known *inverse of $M$,
%\item if an atomic morphism $m$ includes $M\in\Mo{R}{T}$, then $M|_R\equiv M$.
%\end{compactitem}
%
%This is a sound but not complete axiomatization of $M\equiv M'$.
%We can obtain a complete axiomatization in a totally different way: Check $\der_T M(c)\equiv M'(c)$ for all $c\in\dom{S}$.
%However, this can be extremely expensive: It requires computing $\dom{S}$ and proving an object level equality for each identifier.
%Even if the equality of objects is decidable, this is often too expensive.
%
%In fact, because implicit morphisms are meant to be used in very particular cases, we expect the above incomplete calculus to perform reasonably well in practice.
%If showing the equality of certain morphisms is hard, one should probably not make them implicit in the first place.

